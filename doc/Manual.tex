\documentclass[xcolor=dvipsnames,t]{beamer}
% \usepackage[utf8]{inputenc}
\usepackage{verbatim} % for comment
\usetheme{Madrid}
\usecolortheme{seahorse}
\usepackage{beamerouterthemesplit}
\usepackage{array}
\usepackage[T1]{fontenc}


\setbeamerfont{institute}{size=\normalsize}


\title[] {M-SPARC  \\ {\small User guide}}
%\author{Qimen Xu, Abhiraj Sharma, Phanish Suryanarayana}
\institute[]
{
Material Physics \& Mechanics Group, Georgia Tech \\
PI: Phanish Suryanarayana \\
\hyperlink{Contributors}{\texttt{Contributors}} \\
\hyperlink{Citation}{\texttt{Citation}}\\
\hyperlink{Acknowledgments}{\texttt{Acknowledgements}}\\
%Collaborators: J.E. Pask (LLNL), A.J. Medford (GT), E. Chow (GT) \\
%Georgia Institute of Technology
}

\date{}
\setbeamertemplate{frametitle continuation}{}
\setbeamertemplate{navigation symbols}{}
\setbeamercolor{block title}{bg=Apricot!50,fg=black}
%\addtobeamertemplate{block begin}{\vskip -\smallskipamount}{}
%\addtobeamertemplate{block end}{}{\vskip -\smallskipamount}
\addtobeamertemplate{block begin}{\vspace*{-0.6pt}}{}
\addtobeamertemplate{block end}{}{\vspace*{-0.6pt}}
\hypersetup{colorlinks,
citecolor=violet,
linkcolor=blue,
menucolor=white, 
anchorcolor=yellow
filecolor=pink,  
}


\setbeamertemplate{footline}{
    \hbox{%
    \begin{beamercolorbox}[wd=\paperwidth,ht=1ex,dp=1.5ex,leftskip=2ex,rightskip=2ex]{page footer}%
        \usebeamerfont{title in head/foot}%
        \insertshorttitle \hfill
            \insertsection \hfill
        \insertframenumber{} / \inserttotalframenumber
    \end{beamercolorbox}}%
}

% \setbeamertemplate{headline} [split theme] 
% {%
% \begin{beamercolorbox}{section in head/foot}
% \vskip2pt\insertnavigation{\paperwidth}\vskip2pt
% \end{beamercolorbox}%
% }

% \defbeamertemplate*{footline}{split theme}
% {%
%   \leavevmode%
%   \hbox{\begin{beamercolorbox}[wd=.5\paperwidth,ht=2.5ex,dp=1.125ex,leftskip=.3cm plus1fill,rightskip=.3cm]{section in head/foot}%
%     \usebeamerfont{author in head/foot}\insertshortauthor
%   \end{beamercolorbox}%
%   \begin{beamercolorbox}[wd=.5\paperwidth,ht=2.5ex,dp=1.125ex,leftskip=.3cm,rightskip=.3cm plus1fil]{title in head/foot}%
%     \usebeamerfont{title in head/foot}\insertshorttitle
%   \end{beamercolorbox}}%
%   \vskip0pt%
% }



\let\otp\titlepage
\renewcommand{\titlepage}{\otp\addtocounter{framenumber}{-1}}

\begin{document}

%\frame{\titlepage}

\begin{frame}[plain]
 \titlepage
\end{frame}


\begin{frame}[allowframebreaks]{\textbf{Introduction}} \label{Introduction}
Matlab-Simulation Package for Ab-initio Real-space Calculations (M-SPARC) is a real-space code for performing electronic structure calculations based on Kohn-Sham Density Functional Theory (DFT). Its primary purpose is the rapid development and testing of new algorithms and methods within DFT. The main features of M-SPARC include
\begin{itemize}
  \item Applicable to isolated systems such as molecules as well as extended systems such as crystals, surfaces, and wires.
  \item Local (LDA), semilocal (GGA/meta-GGA), and nonlocal (hybrid) exchange-correlation functionals.
  \item Standard ONCV pseudopotentials, including nonlinear core corrections (NLCC).
  \item Calculation of ground state energy, atomic forces, and stress tensor.
  \item Structural relaxation and ab initio molecular dynamics (NVE).
  \item Spin polarized and unpolarized calculations.
  \item Spin-orbit coupling (SOC).
  \item Dispersion interactions through DFT-D3, vdW-DF1, and vdW-DF2.
  \item Soft and transferable table of SPMS pseudopotentials
\end{itemize}
% Additional details regarding the formulation and implementation of M-SPARC can be found in the accompanying paper. Please direct any questions and report any bugs to Prof.~Phanish Suryanarayana.

\end{frame}

\begin{frame}[allowframebreaks]{\textbf{Contributors}} \label{Contributors}
  %\begin{itemize}
      %\item U.S. Department of Energy, Office of Science: DE-SC0019410
      %\item U.S. National Science Foundation: 1333500 and 1553212
  %\end{itemize}

  \begin{itemize}
  \item \textbf{Phanish Suryanarayana} (PI)
  \begin{itemize}
  \item \textbf{Qimen Xu}: Code infrastructure, SCF, Energy, Force, LDA \\  
  \item \textbf{Abhiraj Sharma}: Code infrastructure, k-points, PBE, Stress, Non-orthogonal, Relaxation, NLCC \\   
  \item \textbf{Boqin Zhang}: vdW-DF, DFT-D3, meta-GGA (SCAN) \\  
  \item \textbf{Xin Jing}: Hybrid functionals, SOC \\  
  \item \textbf{Shashikant Kumar}: Testing framework, NLCC \\ 
  \item \textbf{Sayan Bhowmik\footnotemark{}}: DFT+U (Dudarev)
  \end{itemize}
  \item \textbf{Andrew J. Medford} (co-PI)
  \begin{itemize}
      \item \textbf{Sayan Bhowmik\footnotemark[\value{footnote}]}: DFT+U (Dudarev)
  \end{itemize}
  \end{itemize}

\footnotetext{co-advised}
  
\end{frame}


\begin{frame}[allowframebreaks]{\textbf{Citation}} \label{Citation}
If you publish work using/regarding M-SPARC, please cite some of the following articles, particularly those that are most relevant to your work:
\begin{itemize}
    \item General: \url{https://doi.org/10.1016/j.softx.2022.101295 (v2.0.0)}, \url{https://doi.org/10.1016/j.softx.2020.100423 (v1.0.0)}
    \item Non-orthogonal systems: \url{https://doi.org/10.1016/j.cplett.2018.04.018}
    \item Linear solvers: \url{https://doi.org/10.1016/j.cpc.2018.07.007},    \url{https://doi.org/10.1016/j.jcp.2015.11.018}
    \item Stress tensor/pressure: \url{https://doi.org/10.1063/1.5057355}
    \item Atomic forces: \url{https://doi.org/10.1016/j.cpc.2016.09.020}, \url{https://doi.org/10.1016/j.cpc.2017.02.019}
    \item Mixing: \url{https://doi.org/10.1016/j.cplett.2016.01.033}, \url{https://doi.org/10.1016/j.cplett.2015.06.029}, \url{https://doi.org/10.1016/j.cplett.2019.136983}
    \item SPMS pseudopotentials: \url{https://doi.org/10.1016/j.cpc.2022.108594}
\end{itemize}
\end{frame}



  \begin{frame}[allowframebreaks]{\textbf{Acknowledgments}} \label{Acknowledgments}
  
  \begin{itemize}
      \item \textbf{U.S. Department of Energy (DOE), Office of Science (SC): DE-SC0019410} \\

      \vspace{10pt}
      \begin{itemize}
          \item Preliminary developments
          \begin{itemize}
              \item U.S. National Science Foundation: 1333500, and 1553212
          \end{itemize}
      \end{itemize}
      
      
  \end{itemize}
  
  \end{frame}


\begin{frame}[allowframebreaks,fragile]{\textbf{Installation}} \label{Installation}
Prerequisite: \textsc{Matlab}

No installation required.
\end{frame}


\begin{frame}[allowframebreaks]{\textbf{Input files}} \label{Inputfiles}
The required input files to run a simulation with M-SPARC are
\begin{itemize}
  \item ``.inpt" file -- User options and parameters.
  \item ``.ion" file -- Atomic information.
\end{itemize}
It is required that the ``.inpt" and ``.ion" files are located in the same directory and share the same name. A detailed description of the input options is provided in this document. Examples of input files can be found in the directory \texttt{M-SPARC/tests}. \\
In addition, M-SPARC requires pseudopotential files of psp8 format which can be generated by D. R. Hamann's open-source pseudopotential code \href{http://www.mat-simresearch.com/}{ONCVPSP}. A large number of accurate and efficient pseudopotentials are already provided within the package. For access to more pseudopotentials, the user is referred to the \href{http://www.quantum-simulation.org/potentials/sg15_oncv/}{SG15 ONCV potentials}. Using the \href{http://www.mat-simresearch.com/}{ONCVPSP} input files included in the \href{http://www.quantum-simulation.org/potentials/sg15_oncv/}{SG15 ONCV potentials}, one can easily convert the \href{http://www.quantum-simulation.org/potentials/sg15_oncv/}{SG15 ONCV potentials} from upf format to psp8 format. Paths to the pseudopotential files are specified in the ``.ion" file.
\end{frame}



\begin{frame}[allowframebreaks,fragile]{\textbf{Execution}} \label{Execution}
M-SPARC can be executed in MATLAB by calling the \texttt{msparc} function (which is located under \texttt{src/} directory). It is required that the ``.inpt" and ``.ion" files are located in the same directory and share the same name. For example, to run a simulation with input files as ``filename.inpt" and ``filename.ion" in the \texttt{src/} directory, use the following command:
\begin{verbatim}
S = msparc(`filename');
\end{verbatim} 

In many cases, we would not want to put the input files inside the \texttt{src/} directory. In such cases, we need to provide the path to the input file name, without any extension. As an example, one can run a test located in \texttt{M-SPARC/tests/Example\_tests/}. First go to \texttt{src/} directory. Run a DC silicon system  by:
\begin{verbatim}
S = msparc('../tests/Example_tests/Si8_kpt');
\end{verbatim} 
The result is printed to output file "Si8\_kpt.out", located in the same directory as the input files. If the file "Si8\_kpt.out" is already present, the result will be printed to "Si8\_kpt.out\_1" instead. The max number of ".out" files allowed with the same name is 100. Once this number is reached, the result will instead overwrite the "Si8\_kpt.out" file. One can compare the result with the reference out file named "Si8\_kpt.refout".\\~\\

In the \texttt{tests/} directory, we also provide a sample script file \texttt{run\_examples.m}, which launches four example tests one by one. To run these examples, simply change directory to \texttt{tests/examples/} directory, and run: 
\begin{verbatim}
run_examples
\end{verbatim} 
Note that in this case, we're trying to call the \texttt{msparc} function from a different directory. This is achieved by using the MATLAB function \texttt{addpath} to add the \texttt{src/} directory to search path.

One can also run M-SPARC using the MATLAB parallel pool over k-points/spin by providing a second argument, \texttt{parallel\_switch}, when running M-SPARC:
\begin{verbatim}
S = msparc(`filename',parallel_switch);
\end{verbatim} 
If \texttt{parallel\_switch = 1}, M-SPARC will start using the parallel pool, and if \texttt{parallel\_switch = 0}, M-SPARC will not use the parallel pool, which is the default.


A suite of test systems is provided in the \texttt{tests/} directory. The test systems are arranged in a hierarchal systems of directories. Input and reference output files for each test system is stored in separate folders with the same name. A python script named \texttt{'test.py'} is also provided to launch the tests on a cluster. Details on how to use the Python script can be found in \texttt{Readme} file in the \texttt{tests/} folder.
\end{frame}


\begin{frame}[allowframebreaks,fragile]{\textbf{Output}} \label{Output}
Upon successful execution of the  
%\begin{verbatim}
``\texttt{S = msparc(`filename');}"
%\end{verbatim}
command, an output structure is returned and stored in \texttt{S}. The structure \texttt{S} contains detailed information that can be useful for post-processing and debugging. Information such as the input parameters, densities, wavefunctions, eigenvalues, and all electronic ground-state properties calculated are stored in the output structure.\\

Apart from the output structure returned, depending on the calculations performed, some output files will be created in the same location as the input files too. \\

\textbf{Single point calculations}  \\
\begin{itemize}
  \item ``.out" file -- Contains general information about the test, including input parameters, SCF convergence progress, ground state properties and timing information.
  \item ``.static" file -- Contains the atomic positions and atomic forces if the user chooses to print these information..
\end{itemize}

\textbf{Structrual relaxation calculations}  \\
\begin{itemize}
  \item ``.out" file -- See above.
  \item ``.geopt" file -- Contains the atomic positions and atomic forces for each relaxation step. This file is created only when the unit cell is fixed. For cell relaxation a `.cellopt` file is created instead.
  \item ``.cellopt" file -- Contains the cell information (lattice vectors, cell lengths, volume) and stresses for each relaxation step. Only created for cell relaxation.
  \item ``.restart" file -- Contains information necessary to perform a restarted structural relaxation calculation. 
\end{itemize}

\textbf{Molecular dynamics (MD) calculations}  \\
\begin{itemize}
  \item ``.out" file -- See above.
  \item ``.aimd" file -- Contains the atomic positions, atomic velocities, atomic forces, electronic temperature, ionic temperature and total energy for each MD step.
  \item ``.restart" file -- Contains information necessary to perform a restarted MD calculation. 
\end{itemize}

\end{frame}



\begin{frame}[allowframebreaks]{\textbf{Input file options}} \label{Index}
\vspace{-2mm}
\begin{block}{System}
\hyperlink{CELL}{\texttt{CELL}} $\vert$ 
\hyperlink{LATVEC_SCALE}{\texttt{LATVEC\_SCALE}} $\vert$ 
\hyperlink{LATVEC}{\texttt{LATVEC}}  $\vert$ 
\hyperlink{FD_GRID}{\texttt{FD\_GRID}} $\vert$ 
\hyperlink{MESH_SPACING}{\texttt{MESH\_SPACING}} $\vert$ 
\hyperlink{ECUT}{\texttt{ECUT}} $\vert$ 
\hyperlink{BC}{\texttt{BC}} $\vert$ 
\hyperlink{FD_ORDER}{\texttt{FD\_ORDER}} $\vert$ 
\hyperlink{EXCHANGE_CORRELATION}{\texttt{EXCHANGE\_CORRELATION}} $\vert$ 
\hyperlink{SPIN_TYP}{\texttt{SPIN\_TYP}} $\vert$ 
\hyperlink{KPOINT_GRID}{\texttt{KPOINT\_GRID}} $\vert$ 
\hyperlink{KPOINT_SHIFT}{\texttt{KPOINT\_SHIFT}} $\vert$ 
\hyperlink{ELEC_TEMP_TYPE}{\texttt{ELEC\_TEMP\_TYPE}} $\vert$ 
\hyperlink{ELEC_TEMP}{\texttt{ELEC\_TEMP}} $\vert$ 
\hyperlink{SMEARING}{\texttt{SMEARING}} $\vert$ 
\hyperlink{NSTATES}{\texttt{NSTATES}}    $\vert$ 
\hyperlink{D3_FLAG}{\texttt{D3\_FLAG}} $\vert$ 
\hyperlink{D3_RTHR}{\texttt{D3\_RTHR}} $\vert$ 
\hyperlink{D3_CN_THR}{\texttt{D3\_CN\_THR}}$\vert$ 
\hyperlink{VDWDF_GEN_KERNEL}{\texttt{VDWDF\_GEN\_KERNEL}} $\vert$ 
\hyperlink{EXX_RANGE_FOCK}{\texttt{EXX\_RANGE\_FOCK}} $\vert$  
\hyperlink{EXX_RANGE_PBE}{\texttt{EXX\_RANGE\_PBE}} $\vert$ 
\hyperlink{ATOM_TYPE}{\texttt{ATOM\_TYPE}} $\vert$ 
\hyperlink{PSEUDO_POT}{\texttt{PSEUDO\_POT}}  $\vert$ 
\hyperlink{N_TYPE_ATOM}{\texttt{N\_TYPE\_ATOM}} $\vert$ 
\hyperlink{COORD}{\texttt{COORD}} $\vert$ 
\hyperlink{COORD_FRAC}{\texttt{COORD\_FRAC}} $\vert$ 
\hyperlink{RELAX}{\texttt{RELAX}} $\vert$ 
\hyperlink{SPIN}{\texttt{SPIN}} $\vert$ 
\hyperlink{HUBBARD}{\texttt{HUBBARD}} $\vert$ 
\hyperlink{U_ATOM_TYPE}{\texttt{U\_ATOM\_TYPE}} $\vert$ 
\hyperlink{U_VAL}{\texttt{U\_VAL}} $\vert$ 
\hyperlink{CORE_FLAG}{\texttt{CORE\_FLAG}} 
\end{block}

\begin{block}{SCF}
\hyperlink{CHEB_DEGREE}{\texttt{CHEB\_DEGREE}} $\vert$ 
\hyperlink{RHO_TRIGGER}{\texttt{RHO\_TRIGGER}} $\vert$ 
\hyperlink{NUM_CHEFSI}{\texttt{NUM\_CHEFSI}} $\vert$ 
\hyperlink{MAXIT_SCF}{\texttt{MAXIT\_SCF}} $\vert$ 
\hyperlink{TOL_SCF}{\texttt{TOL\_SCF}} $\vert$ 
\hyperlink{SCF_FORCE_ACC}{\texttt{SCF\_FORCE\_ACC}} $\vert$ 
\hyperlink{SCF_ENERGY_ACC}{\texttt{SCF\_ENERGY\_ACC}} $\vert$ 
\hyperlink{TOL_LANCZOS}{\texttt{TOL\_LANCZOS}} $\vert$ 
\hyperlink{MIXING_VARIABLE}{\texttt{MIXING\_VARIABLE}} $\vert$ 
\hyperlink{MIXING_HISTORY}{\texttt{MIXING\_HISTORY}} $\vert$ 
\hyperlink{MIXING_PARAMETER}{\texttt{MIXING\_PARAMETER}} $\vert$ 
\hyperlink{MIXING_PARAMETER_SIMPLE}{\texttt{MIXING\_PARAMETER\_SIMPLE}} $\vert$ 
\hyperlink{MIXING_PARAMETER_MAG}{\texttt{MIXING\_PARAMETER\_MAG}} $\vert$
\hyperlink{MIXING_PARAMETER_SIMPLE_MAG}{\texttt{MIXING\_PARAMETER\_SIMPLE\_MAG}} $\vert$
\hyperlink{PULAY_FREQUENCY}{\texttt{PULAY\_FREQUENCY}} $\vert$ 
\hyperlink{PULAY_RESTART}{\texttt{PULAY\_RESTART}} $\vert$ 
\hyperlink{MIXING_PRECOND}{\texttt{MIXING\_PRECOND}} $\vert$ 
\hyperlink{MIXING_PRECOND_MAG}{\texttt{MIXING\_PRECOND\_MAG}} $\vert$ 
\hyperlink{TOL_PRECOND}{\texttt{TOL\_PRECOND}} $\vert$ 
\hyperlink{PRECOND_KERKER_KTF}{\texttt{PRECOND\_KERKER\_KTF}} $\vert$ 
\hyperlink{PRECOND_KERKER_THRESH}{\texttt{PRECOND\_KERKER\_THRESH}} $\vert$ 
\hyperlink{PRECOND_KERKER_KTF_MAG}{\texttt{PRECOND\_KERKER\_KTF\_MAG}} $\vert$ 
\hyperlink{PRECOND_KERKER_THRESH\_MAG}{\texttt{PRECOND\_KERKER\_THRESH\_MAG}} $\vert$ 
% \hyperlink{PRECOND_RESTA_Q0}{\texttt{PRECOND\_RESTA\_Q0}} $\vert$ 
% \hyperlink{PRECOND_RESTA_RS}{\texttt{PRECOND\_RESTA\_RS}} $\vert$ 
% \hyperlink{PRECOND_FITPOW}{\texttt{PRECOND\_FITPOW}} $\vert$ 
\hyperlink{TOL_FOCK}{\texttt{TOL\_FOCK}} $\vert$ 
\hyperlink{MAXIT_FOCK}{\texttt{MAXIT\_FOCK}} $\vert$ 
\hyperlink{MINIT_FOCK}{\texttt{MINIT\_FOCK}} $\vert$ 
\hyperlink{TOL_SCF_INIT}{\texttt{TOL\_SCF\_INIT}} $\vert$ 
\hyperlink{ACE_FLAG}{\texttt{ACE\_FLAG}} $\vert$ 
\hyperlink{EXX_METHOD}{\texttt{EXX\_METHOD}} $\vert$ 
\hyperlink{EXX_ACE_VALENCE_STATES}{\texttt{EXX\_ACE\_VALENCE\_STATES}} $\vert$ 
\hyperlink{EXX_DOWNSAMPLING}{\texttt{EXX\_DOWNSAMPLING}} $\vert$ 
\hyperlink{EXX_DIVERGENCE}{\texttt{EXX\_DIVERGENCE}}
\end{block}

\vspace{-2mm}
\begin{block}{Electrostatics}
\hyperlink{TOL_POISSON}{\texttt{TOL\_POISSON}} $\vert$ \hyperlink{TOL_PSEUDOCHARGE}{\texttt{TOL\_PSEUDOCHARGE}} $\vert$ \hyperlink{REFERENCE_CUTOFF}{\texttt{REFERENCE\_CUTOFF}}
\end{block}

\vspace{-2mm}
\begin{block}{Stress calculation}
\hyperlink{CALC_STRESS}{\texttt{CALC\_STRESS}} $\vert$ \hyperlink{CALC_PRES}{\texttt{CALC\_PRES}}
\end{block}
\vspace{-2mm}

\begin{block}{MD}
\hyperlink{MD_FLAG}{\texttt{MD\_FLAG}} $\vert$ \hyperlink{MD_METHOD}{\texttt{MD\_METHOD}} $\vert$ \hyperlink{MD_NSTEP}{\texttt{MD\_NSTEP}} $\vert$ \hyperlink{MD_TIMESTEP}{\texttt{MD\_TIMESTEP}} $\vert$ \hyperlink{ION_TEMP}{\texttt{ION\_TEMP}} $\vert$ \hyperlink{ION_ELEC_EQT}{\texttt{ION\_ELEC\_EQT}} $\vert$ \hyperlink{RESTART_FLAG}{\texttt{RESTART\_FLAG}}
\end{block}
\vspace{-2mm}
\begin{block}{Structural relaxation}
\hyperlink{RELAX_FLAG}{\texttt{RELAX\_FLAG}} $\vert$ \hyperlink{RELAX_METHOD}{\texttt{RELAX\_METHOD}} $\vert$ \hyperlink{RELAX_NITER}{\texttt{RELAX\_NITER}} $\vert$ \hyperlink{TOL_RELAX}{\texttt{TOL\_RELAX}} $\vert$ \hyperlink{TOL_RELAX_CELL}{\texttt{TOL\_RELAX\_CELL}} $\vert$ \hyperlink{RELAX_MAXDIAL}{\texttt{RELAX\_MAXDIAL}} $\vert$ \hyperlink{NLCG_SIGMA}{\texttt{NLCG\_SIGMA}} $\vert$ \hyperlink{L_HISTORY}{\texttt{L\_HISTORY}} $\vert$ \hyperlink{L_FINIT_STP}{\texttt{L\_FINIT\_STP}} $\vert$ \hyperlink{L_MAXMOV}{\texttt{L\_MAXMOV}} $\vert$ \hyperlink{L_AUTOSCALE}{\texttt{L\_AUTOSCALE}} $\vert$ \hyperlink{L_LINEOPT}{\texttt{L\_LINEOPT}} $\vert$ \hyperlink{L_ICURV}{\texttt{L\_ICURV}} $\vert$ \hyperlink{FIRE_DT}{\texttt{FIRE\_DT}} $\vert$ \hyperlink{FIRE_MASS}{\texttt{FIRE\_MASS}} $\vert$ \hyperlink{FIRE_MAXMOV}{\texttt{FIRE\_MAXMOV}} $\vert$ \hyperlink{RESTART_FLAG}{\texttt{RESTART\_FLAG}}
\end{block}

\begin{block}{Print options}
\hyperlink{PRINT_ATOMS}{\texttt{PRINT\_ATOMS}} $\vert$ \hyperlink{PRINT_FORCES}{\texttt{PRINT\_FORCES}} $\vert$ \hyperlink{PRINT_MDOUT}{\texttt{PRINT\_MDOUT}} $\vert$ \hyperlink{PRINT_RELAXOUT}{\texttt{PRINT\_RELAXOUT}} $\vert$ \hyperlink{PRINT_RESTART}{\texttt{PRINT\_RESTART}} $\vert$ \hyperlink{PRINT_RESTART_FQ}{\texttt{PRINT\_RESTART\_FQ}} $\vert$ \hyperlink{PRINT_VELS}{\texttt{PRINT\_VELS}} $\vert$ \hyperlink{OUTPUT_FILE}{\texttt{OUTPUT\_FILE}}
\end{block}


\end{frame}



%%%%%%%%%%%%%%%%%%%%%%%%%%%%%%%%%%%%%%%%%%%%%%%%%%%%%%%%%%%%%%%%%%%%%%%%%%%%%%%%%%%%%%%%%%%%%
\begin{frame}[allowframebreaks,c]{} \label{System}

\begin{center}
\Huge \textbf{System: .inpt file}
\end{center}

\end{frame}
%%%%%%%%%%%%%%%%%%%%%%%%%%%%%%%%%%%%%%%%%%%%%%%%%%%%%%%%%%%%%%%%%%%%%%%%%%%%%%%%%%%%%%%%%%%%%



%%%%%%%%%%%%%%%%%%%%%%%%%%%%%%%%%%%%%%%%%%%%%%%%%%%%%%%%%%%%%%%%%%%%%%%%%%%%%%%%%%%%%%%%%%%%%
\begin{frame}[allowframebreaks]{\texttt{CELL}} \label{CELL}
\vspace*{-12pt}
\begin{columns}
\column{0.4\linewidth}
\begin{block}{Type}
Double
\end{block}

\begin{block}{Default}
None
\end{block}

\column{0.4\linewidth}
\begin{block}{Unit}
Bohr
\end{block}
    
\begin{block}{Example}
\texttt{CELL}: 10.20 11.21 7.58
\end{block}
\end{columns}
\begin{block}{Description}
A set of three whitespace delimited values specifying the cell lengths in the lattice vector (\hyperlink{LATVEC}{\texttt{LATVEC}}) directions, respectively.
\end{block}
\begin{block}{Remark}
    Note that \hyperlink{CELL}{\texttt{CELL}} ignores the lengths of the lattice vectors specified in the \texttt{.inpt} file and only treats them as unit vectors. \hyperlink{LATVEC_SCALE}{\texttt{LATVEC\_SCALE}} and \hyperlink{CELL}{\texttt{CELL}} cannot be specified simultaneously.
\end{block}

\end{frame}
%%%%%%%%%%%%%%%%%%%%%%%%%%%%%%%%%%%%%%%%%%%%%%%%%%%%%%%%%%%%%%%%%%%%%%%%%%%%%%%%%%%%%%%%%%%%%


%%%%%%%%%%%%%%%%%%%%%%%%%%%%%%%%%%%%%%%%%%%%%%%%%%%%%%%%%%%%%%%%%%%%%%%%%%%%%%%%%%%%%%%%%%%%%
\begin{frame}[allowframebreaks]{\texttt{LATVEC\_SCALE}} \label{LATVEC_SCALE}
\vspace*{-15pt}
\begin{columns}
\column{0.4\linewidth}
\begin{block}{Type}
Double
\end{block}

\begin{block}{Default}
None
\end{block}

\column{0.5\linewidth}
\begin{block}{Unit}
Bohr
\end{block}

\begin{block}{Example}
\texttt{LATVEC\_SCALE}: 10.20 11.21 7.58
\end{block}
\end{columns}

\vspace*{-2pt}
\begin{block}{Description}
A set of three whitespace delimited values specifying the scaling factors in the lattice vectors (\hyperlink{LATVEC}{\texttt{LATVEC}}), respectively.
\end{block}
\vspace*{-4pt}
\begin{block}{Remark}
    The difference between \hyperlink{LATVEC_SCALE}{\texttt{LATVEC\_SCALE}} and \hyperlink{CELL}{\texttt{CELL}} is that \hyperlink{CELL}{\texttt{CELL}} treats the lattice vectors as unit vectors, whereas \hyperlink{LATVEC_SCALE}{\texttt{LATVEC\_SCALE}} scales the lattice vectors directly as specified by the user. \hyperlink{LATVEC_SCALE}{\texttt{LATVEC\_SCALE}} and \hyperlink{CELL}{\texttt{CELL}} cannot be specified simultaneously.
\end{block}

\end{frame}
%%%%%%%%%%%%%%%%%%%%%%%%%%%%%%%%%%%%%%%%%%%%%%%%%%%%%%%%%%%%%%%%%%%%%%%%%%%%%%%%%%%%%%%%%%%%%



%%%%%%%%%%%%%%%%%%%%%%%%%%%%%%%%%%%%%%%%%%%%%%%%%%%%%%%%%%%%%%%%%%%%%%%%%%%%%%%%%%%%%%%%%%%%%
\begin{frame}[allowframebreaks]{\texttt{LATVEC}} \label{LATVEC}
\vspace*{-12pt}
\begin{columns}
\column{0.4\linewidth}
\begin{block}{Type}
Double array
\end{block}

\begin{block}{Default}
1.0 0.0 0.0\\
0.0 1.0 0.0\\
0.0 0.0 1.0\\
\end{block}

\column{0.4\linewidth}
\begin{block}{Unit}
No unit
\end{block}

\begin{block}{Example}
\texttt{LATVEC}: \\
0.5 0.5 0.0\\
0.0 0.5 0.5\\
0.5 0.0 0.5\\
\end{block}
\end{columns}

\begin{block}{Description}
A set of three vectors in row major order specifying the lattice vectors of the simulation domain (\hyperlink{CELL}{\texttt{CELL}}).
\end{block}

%\begin{block}{Remark}
%Lattice vectors need not be unit vectors.
%\end{block}

\end{frame}
%%%%%%%%%%%%%%%%%%%%%%%%%%%%%%%%%%%%%%%%%%%%%%%%%%%%%%%%%%%%%%%%%%%%%%%%%%%%%%%%%%%%%%%%%%%%%


%%%%%%%%%%%%%%%%%%%%%%%%%%%%%%%%%%%%%%%%%%%%%%%%%%%%%%%%%%%%%%%%%%%%%%%%%%%%%%%%%%%%%%%%%%%%%
\begin{frame}[allowframebreaks]{\texttt{FD\_GRID}} \label{FD_GRID}
\vspace*{-12pt}
\begin{columns}
\column{0.4\linewidth}
\begin{block}{Type}
Integer
\end{block}

\begin{block}{Default}
None
\end{block}

\column{0.4\linewidth}
\begin{block}{Unit}
No unit
\end{block}

\begin{block}{Example}
\texttt{FD\_GRID}: 26 26 30
\end{block}
\end{columns}

\begin{block}{Description}
A set of three whitespace delimited values specifying the number of finite-difference intervals in the lattice vector (\hyperlink{LATVEC}{\texttt{LATVEC}}) directions, respectively.
\end{block}

\begin{block}{Remark}
The convergence of results with respect to spatial discretization needs to be verified. \hyperlink{ECUT}{\texttt{ECUT}}, \hyperlink{MESH_SPACING}{\texttt{MESH\_SPACING}}, \hyperlink{FD_GRID}{\texttt{FD\_GRID}} cannot be specified simultaneously.
\end{block}

\end{frame}
%%%%%%%%%%%%%%%%%%%%%%%%%%%%%%%%%%%%%%%%%%%%%%%%%%%%%%%%%%%%%%%%%%%%%%%%%%%%%%%%%%%%%%%%%%%%%




%%%%%%%%%%%%%%%%%%%%%%%%%%%%%%%%%%%%%%%%%%%%%%%%%%%%%%%%%%%%%%%%%%%%%%%%%%%%%%%%%%%%%%%%%%%%%
\begin{frame}[allowframebreaks]{\texttt{MESH\_SPACING}} \label{MESH_SPACING}
\vspace*{-12pt}
\begin{columns}
\column{0.4\linewidth}
\begin{block}{Type}
Double
\end{block}

\begin{block}{Default}
None
\end{block}

\column{0.4\linewidth}
\begin{block}{Unit}
Bohr
\end{block}

\begin{block}{Example}
\texttt{MESH\_SPACING}: 0.4
\end{block}
\end{columns}

\begin{block}{Description}
Mesh spacing of the finite-difference grid. 
\end{block}

\begin{block}{Remark}
The exact mesh-size will be determined by the size of the domain.  \hyperlink{MESH_SPACING}{\texttt{MESH\_SPACING}}, \hyperlink{FD_GRID}{\texttt{FD\_GRID}}, \hyperlink{ECUT}{\texttt{ECUT}} cannot be specified simultaneously.
\end{block}

\end{frame}
%%%%%%%%%%%%%%%%%%%%%%%%%%%%%%%%%%%%%%%%%%%%%%%%%%%%%%%%%%%%%%%%%%%%%%%%%%%%%%%%%%%%%%%%%%%%%




%%%%%%%%%%%%%%%%%%%%%%%%%%%%%%%%%%%%%%%%%%%%%%%%%%%%%%%%%%%%%%%%%%%%%%%%%%%%%%%%%%%%%%%%%%%%%
\begin{frame}[allowframebreaks]{\texttt{ECUT}} \label{ECUT}
\vspace*{-12pt}
\begin{columns}
\column{0.4\linewidth}
\begin{block}{Type}
Double
\end{block}

\begin{block}{Default}
None
\end{block}

\column{0.4\linewidth}
\begin{block}{Unit}
Ha
\end{block}

\begin{block}{Example}
\texttt{ECUT}: 30
\end{block}
\end{columns}

\begin{block}{Description}
Equivalent plane-wave energy cutoff, based on which \hyperlink{MESH_SPACING}{\texttt{MESH\_SPACING}} will be automatically calculated. 
\end{block}

\begin{block}{Remark}
This is not exact, but rather an estimate. \hyperlink{ECUT}{\texttt{ECUT}}, \hyperlink{MESH_SPACING}{\texttt{MESH\_SPACING}}, \hyperlink{FD_GRID}{\texttt{FD\_GRID}} cannot be specified simultaneously.
\end{block}

\end{frame}
%%%%%%%%%%%%%%%%%%%%%%%%%%%%%%%%%%%%%%%%%%%%%%%%%%%%%%%%%%%%%%%%%%%%%%%%%%%%%%%%%%%%%%%%%%%%%





%%%%%%%%%%%%%%%%%%%%%%%%%%%%%%%%%%%%%%%%%%%%%%%%%%%%%%%%%%%%%%%%%%%%%%%%%%%%%%%%%%%%%%%%%%%%%
\begin{frame}[allowframebreaks]{\texttt{BC}} \label{BC}
\vspace*{-12pt}
\begin{columns}
\column{0.4\linewidth}
\begin{block}{Type}
Character
\end{block}

\begin{block}{Default}
None
\end{block}

\column{0.4\linewidth}
\begin{block}{Unit}
No unit
\end{block}

\begin{block}{Example}
\texttt{BC}: \texttt{P D D}
\end{block}
\end{columns}

\begin{block}{Description}
A set of three whitespace delimited characters specifying the boundary conditions in the lattice vector directions, respectively. \texttt{P} represents periodic boundary conditions and \texttt{D} represents Dirichlet boundary conditions.
\end{block}

\end{frame}
%%%%%%%%%%%%%%%%%%%%%%%%%%%%%%%%%%%%%%%%%%%%%%%%%%%%%%%%%%%%%%%%%%%%%%%%%%%%%%%%%%%%%%%%%%%%%



%%%%%%%%%%%%%%%%%%%%%%%%%%%%%%%%%%%%%%%%%%%%%%%%%%%%%%%%%%%%%%%%%%%%%%%%%%%%%%%%%%%%%%%%%%%%%
\begin{frame}[allowframebreaks]{\texttt{FD\_ORDER}} \label{FD_ORDER}
\vspace*{-12pt}
\begin{columns}
\column{0.4\linewidth}
\begin{block}{Type}
Integer
\end{block}

\begin{block}{Default}
12
\end{block}

\column{0.4\linewidth}
\begin{block}{Unit}
No unit
\end{block}

\begin{block}{Example}
\texttt{FD\_ORDER}: 12
\end{block}
\end{columns}

\begin{block}{Description}
Order of the finite-difference approximation. 
\end{block}

\begin{block}{Remark}
Restricted to even integers since central finite-differences are employed. The default value of 12 has been found to be an efficient choice for most systems.
\end{block}

\end{frame}
%%%%%%%%%%%%%%%%%%%%%%%%%%%%%%%%%%%%%%%%%%%%%%%%%%%%%%%%%%%%%%%%%%%%%%%%%%%%%%%%%%%%%%%%%%%%%



%%%%%%%%%%%%%%%%%%%%%%%%%%%%%%%%%%%%%%%%%%%%%%%%%%%%%%%%%%%%%%%%%%%%%%%%%%%%%%%%%%%%%%%%%%%%%
\begin{frame}[allowframebreaks]{\texttt{EXCHANGE\_CORRELATION}} \label{EXCHANGE_CORRELATION}
\vspace*{-12pt}
\begin{columns}
\column{0.35\linewidth}
\begin{block}{Type}
String
\end{block}

\begin{block}{Default}
No Default
\end{block}

\column{0.55\linewidth}
\begin{block}{Unit}
No unit
\end{block}

\begin{block}{Example}
\texttt{EXCHANGE\_CORRELATION}: \texttt{LDA\_PW}
\end{block}
\end{columns}

\begin{block}{Description}
  Choice of exchange-correlation functional. Options are \texttt{LDA\_PW} (Perdew-Wang LDA), \texttt{LDA\_PZ} (Purdew-Zunger LDA), \texttt{GGA\_PBE} (PBE GGA), \texttt{GGA\_RPBE} (PBE RGGA), \texttt{GGA\_PBEsol} (PBE GGAsol), \texttt{vdWDF1} (van der Waals Density Functional developed by Dion et al.), \texttt{vdWDF2} (vdW Density Functional modified by Lee et al), \texttt{SCAN} (SCAN metaGGA), \texttt{RSCAN} (rSCAN metaGGA) and \texttt{R2SCAN} (r2SCAN metaGGA).
\end{block}

\begin{block}{Remark}
For spin-polarized calculation (\hyperlink{SPIN_TYP}{\texttt{SPIN\_TYP}} = 1), \texttt{LDA\_PZ} is not available.

Before using \texttt{vdWDF1} or \texttt{vdWDF2}, please read the description and remark of \hyperlink{VDWDF_GEN_KERNEL}{\texttt{VDWDF\_GEN\_KERNEL}}.

Currently SCAN and R2SCAN does not support nonlinear core correction pseudopotential.
\end{block}

\end{frame}
%%%%%%%%%%%%%%%%%%%%%%%%%%%%%%%%%%%%%%%%%%%%%%%%%%%%%%%%%%%%%%%%%%%%%%%%%%%%%%%%%%%%%%%%%%%%%




%%%%%%%%%%%%%%%%%%%%%%%%%%%%%%%%%%%%%%%%%%%%%%%%%%%%%%%%%%%%%%%%%%%%%%%%%%%%%%%%%%%%%%%%%%%%%
\begin{frame}[allowframebreaks]{\texttt{SPIN\_TYP}} \label{SPIN_TYP}
\vspace*{-12pt}
\begin{columns}
\column{0.4\linewidth}
\begin{block}{Type}
Integer
\end{block}

\begin{block}{Default}
0
\end{block}

\column{0.4\linewidth}
\begin{block}{Unit}
No unit
\end{block}

\begin{block}{Example}
\texttt{SPIN\_TYP}: 1
\end{block}
\end{columns}

\begin{block}{Description}
\texttt{SPIN\_TYP}: 0 performs spin unpolarized calculation. \\
\texttt{SPIN\_TYP}: 1 performs unconstrained collinear spin-polarized calculation.  \\
\texttt{SPIN\_TYP}: 1 performs unconstrained noncollinear spin-polarized calculation.   
\end{block}

\begin{block}{Remark}
\texttt{SPIN\_TYP} can only take values 0, 1, 2. For collinear calculation, non-relativistic pseudopotential need to be used. For noncollinear calculation, fully relativistic pseudopotentiail need to be used. 
\end{block}

\end{frame}
%%%%%%%%%%%%%%%%%%%%%%%%%%%%%%%%%%%%%%%%%%%%%%%%%%%%%%%%%%%%%%%%%%%%%%%%%%%%%%%%%%%%%%%%%%%%%



%%%%%%%%%%%%%%%%%%%%%%%%%%%%%%%%%%%%%%%%%%%%%%%%%%%%%%%%%%%%%%%%%%%%%%%%%%%%%%%%%%%%%%%%%%%%%
\begin{frame}[allowframebreaks]{\texttt{KPOINT\_GRID}} \label{KPOINT_GRID}
\vspace*{-12pt}
\begin{columns}
\column{0.4\linewidth}
\begin{block}{Type}
Integer array
\end{block}

\begin{block}{Default}
 1 1 1
\end{block}

\column{0.4\linewidth}
\begin{block}{Unit}
No unit
\end{block}

\begin{block}{Example}
\texttt{KPOINT\_GRID}: 2 3 4
\end{block}
\end{columns}

\begin{block}{Description}
Number of k-points in each direction of the Monkhorst-Pack grid for Brillouin zone integration.
\end{block}

\begin{block}{Remark}
Time-reversal symmetry is assumed to hold. 
\end{block}

\end{frame}
%%%%%%%%%%%%%%%%%%%%%%%%%%%%%%%%%%%%%%%%%%%%%%%%%%%%%%%%%%%%%%%%%%%%%%%%%%%%%%%%%%%%%%%%%%%%%


%%%%%%%%%%%%%%%%%%%%%%%%%%%%%%%%%%%%%%%%%%%%%%%%%%%%%%%%%%%%%%%%%%%%%%%%%%%%%%%%%%%%%%%%%%%%%
\begin{frame}[allowframebreaks]{\texttt{KPOINT\_SHIFT}} \label{KPOINT_SHIFT}
\vspace*{-12pt}
\begin{columns}
\column{0.4\linewidth}
\begin{block}{Type}
Double array
\end{block}

\begin{block}{Default}
 0.0 0.0 0.0
\end{block}

\column{0.4\linewidth}
\begin{block}{Unit}
No unit
\end{block}

\begin{block}{Example}
\texttt{KPOINT\_SHIFT}: 0.5 0.5 0.5
\end{block}
\end{columns}

\begin{block}{Description}
Shift of k-points in each direction of the k-point lattice.
\end{block}

\begin{block}{Remark}
The shift is in reduced coordinates. The default zero shift corresponds to the Monkhorst-Pack grid.
\end{block}

\end{frame}
%%%%%%%%%%%%%%%%%%%%%%%%%%%%%%%%%%%%%%%%%%%%%%%%%%%%%%%%%%%%%%%%%%%%%%%%%%%%%%%%%%%%%%%%%%%%%


%%%%%%%%%%%%%%%%%%%%%%%%%%%%%%%%%%%%%%%%%%%%%%%%%%%%%%%%%%%%%%%%%%%%%%%%%%%%%%%%%%%%%%%%%%%%%
\begin{frame}[allowframebreaks]{\texttt{ELEC\_TEMP\_TYPE}} \label{ELEC_TEMP_TYPE}
\vspace*{-12pt}
\begin{columns}
\column{0.4\linewidth}
\begin{block}{Type}
String
\end{block}

\begin{block}{Default}
\texttt{gaussian}
\end{block}

\column{0.4\linewidth}
\begin{block}{Unit}
No unit
\end{block}

\begin{block}{Example}
\texttt{ELEC\_TEMP\_TYPE}: \texttt{fd}
\end{block}
\end{columns}

\begin{block}{Description}
Function used for the smearing (electronic temperature). Options are: \texttt{fermi-dirac} (or \texttt{fd}), \texttt{gaussian}.
\end{block}

\begin{block}{Remark}
Use \hyperlink{ELEC_TEMP}{\texttt{ELEC\_TEMP}} or \hyperlink{SMEARING}{\texttt{SMEARING}} to set smearing value.
\end{block}

\end{frame}
%%%%%%%%%%%%%%%%%%%%%%%%%%%%%%%%%%%%%%%%%%%%%%%%%%%%%%%%%%%%%%%%%%%%%%%%%%%%%%%%%%%%%%%%%%%%%


%%%%%%%%%%%%%%%%%%%%%%%%%%%%%%%%%%%%%%%%%%%%%%%%%%%%%%%%%%%%%%%%%%%%%%%%%%%%%%%%%%%%%%%%%%%%%
\begin{frame}[allowframebreaks]{\texttt{ELEC\_TEMP}} \label{ELEC_TEMP}
\vspace*{-12pt}
\begin{columns}
\column{0.4\linewidth}
\begin{block}{Type}
Double
\end{block}

\begin{block}{Default}
1160.452
\end{block}

\column{0.4\linewidth}
\begin{block}{Unit}
Kelvin
\end{block}

\begin{block}{Example}
\texttt{ELEC\_TEMP}: 315.773
\end{block}
\end{columns}

\begin{block}{Description}
Electronic temperature. 
\end{block}

\begin{block}{Remark}
This is equivalent to setting \hyperlink{SMEARING}{\texttt{SMEARING}} (0.001 Ha = 315.773 Kelvin).
\end{block}

\end{frame}
%%%%%%%%%%%%%%%%%%%%%%%%%%%%%%%%%%%%%%%%%%%%%%%%%%%%%%%%%%%%%%%%%%%%%%%%%%%%%%%%%%%%%%%%%%%%%



%%%%%%%%%%%%%%%%%%%%%%%%%%%%%%%%%%%%%%%%%%%%%%%%%%%%%%%%%%%%%%%%%%%%%%%%%%%%%%%%%%%%%%%%%%%%%
\begin{frame}[allowframebreaks]{\texttt{SMEARING}} \label{SMEARING}
\vspace*{-12pt}
\begin{columns}
\column{0.4\linewidth}
\begin{block}{Type}
Double
\end{block}

\begin{block}{Default}
0.003675 for \texttt{gaussian} \\
0.007350 for \texttt{fermi-dirac}
\end{block}

\column{0.4\linewidth}
\begin{block}{Unit}
Ha
\end{block}

\begin{block}{Example}
\texttt{SMEARING}: 0.001
\end{block}
\end{columns}

\begin{block}{Description}
Value of smearing.
\end{block}

\begin{block}{Remark}
Equivalent to setting \hyperlink{ELEC_TEMP}{\texttt{ELEC\_TEMP}} (0.001 Ha = 315.773 Kelvin).
\end{block}

\end{frame}
%%%%%%%%%%%%%%%%%%%%%%%%%%%%%%%%%%%%%%%%%%%%%%%%%%%%%%%%%%%%%%%%%%%%%%%%%%%%%%%%%%%%%%%%%%%%%




%%%%%%%%%%%%%%%%%%%%%%%%%%%%%%%%%%%%%%%%%%%%%%%%%%%%%%%%%%%%%%%%%%%%%%%%%%%%%%%%%%%%%%%%%%%%%
\begin{frame}[allowframebreaks]{\texttt{NSTATES}} \label{NSTATES}
\vspace*{-12pt}
\begin{columns}
\column{0.4\linewidth}
\begin{block}{Type}
Integer
\end{block}

\begin{block}{Default}
$N_e/2 \times 1.2 + 5$
\end{block}

\column{0.4\linewidth}
\begin{block}{Unit}
No unit
\end{block}

\begin{block}{Example}
\texttt{NSTATES}: 24
\end{block}
\end{columns}

\begin{block}{Description}
The number of Kohn-Sham states/orbitals.
\end{block}

\begin{block}{Remark}
This number should not be smaller than half of
the total number of valence electrons ($N_e$) in the system. Note that the number of additional states required increases with increasing values of \hyperlink{ELEC_TEMP}{\texttt{ELEC\_TEMP}}/\hyperlink{SMEARING}{\texttt{SMEARING}}.
\end{block}

\end{frame}
%%%%%%%%%%%%%%%%%%%%%%%%%%%%%%%%%%%%%%%%%%%%%%%%%%%%%%%%%%%%%%%%%%%%%%%%%%%%%%%%%%%%%%%%%%%%%


%%%%%%%%%%%%%%%%%%%%%%%%%%%%%%%%%%%%%%%%%%%%%%%%%%%%%%%%%%%%%%%%%%%%%%%%%%%%%%%%%%%%%%%%%%%%%
\begin{frame}[allowframebreaks]{\texttt{D3\_FLAG}} \label{D3_FLAG}
\vspace*{-12pt}
\begin{columns}
\column{0.4\linewidth}
\begin{block}{Type}
0 or 1
\end{block}

\begin{block}{Default}
\texttt{0}
\end{block}

\column{0.4\linewidth}
\begin{block}{Unit}
No unit
\end{block}

\begin{block}{Example}
\texttt{D3\_FLAG}: \texttt{1}
\end{block}
\end{columns}

\begin{block}{Description}
Flag for adding Grimme’s DFT-D3 correction on the result
\end{block}

\begin{block}{Remark}
Only active when using GGA-PBE, GGA-RPBE and GGA-PBEsol.
\end{block}

\end{frame}
%%%%%%%%%%%%%%%%%%%%%%%%%%%%%%%%%%%%%%%%%%%%%%%%%%%%%%%%%%%%%%%%%%%%%%%%%%%%%%%%%%%%%%%%%%%%%


%%%%%%%%%%%%%%%%%%%%%%%%%%%%%%%%%%%%%%%%%%%%%%%%%%%%%%%%%%%%%%%%%%%%%%%%%%%%%%%%%%%%%%%%%%%%%
\begin{frame}[allowframebreaks]{\texttt{D3\_RTHR}} \label{D3_RTHR}
\vspace*{-12pt}
\begin{columns}
\column{0.4\linewidth}
\begin{block}{Type}
Double
\end{block}

\begin{block}{Default}
\texttt{1600}
\end{block}

\column{0.4\linewidth}
\begin{block}{Unit}
Bohr$^2$
\end{block}

\begin{block}{Example}
\texttt{D3\_RTHR}: \texttt{9000}
\end{block}
\end{columns}

\begin{block}{Description}
Square of cut-off radius for calculating DFT-D3 correction between two atoms
\end{block}

\begin{block}{Remark}
Only applicable when DFT-D3 correction \hyperlink{D3_FLAG}{\texttt{D3\_FLAG}} is used.
\end{block}

\end{frame}
%%%%%%%%%%%%%%%%%%%%%%%%%%%%%%%%%%%%%%%%%%%%%%%%%%%%%%%%%%%%%%%%%%%%%%%%%%%%%%%%%%%%%%%%%%%%%


%%%%%%%%%%%%%%%%%%%%%%%%%%%%%%%%%%%%%%%%%%%%%%%%%%%%%%%%%%%%%%%%%%%%%%%%%%%%%%%%%%%%%%%%%%%%%
\begin{frame}[allowframebreaks]{\texttt{D3\_CN\_THR}} \label{D3_CN_THR}
\vspace*{-12pt}
\begin{columns}
\column{0.4\linewidth}
\begin{block}{Type}
Double
\end{block}

\begin{block}{Default}
\texttt{625}
\end{block}

\column{0.4\linewidth}
\begin{block}{Unit}
Bohr$^2$
\end{block}

\begin{block}{Example}
\texttt{D3\_CN\_THR}: \texttt{1600}
\end{block}
\end{columns}

\begin{block}{Description}
Square of cut-off radius for calculating CN value of every atom and DFT-D3 correction between three atoms
\end{block}

\begin{block}{Remark}
Only applicable when DFT-D3 correction \hyperlink{D3_FLAG}{\texttt{D3\_FLAG}} is used.
\end{block}

\end{frame}
%%%%%%%%%%%%%%%%%%%%%%%%%%%%%%%%%%%%%%%%%%%%%%%%%%%%%%%%%%%%%%%%%%%%%%%%%%%%%%%%%%%%%%%%%%%%%


%%%%%%%%%%%%%%%%%%%%%%%%%%%%%%%%%%%%%%%%%%%%%%%%%%%%%%%%%%%%%%%%%%%%%%%%%%%%%%%%%%%%%%%%%%%%%
\begin{frame}[allowframebreaks]{\texttt{VDWDF\_GEN\_KERNEL}} \label{VDWDF_GEN_KERNEL}
\vspace*{-12pt}
\begin{columns}
\column{0.4\linewidth}
\begin{block}{Type}
0 or 1
\end{block}

\begin{block}{Default}
\texttt{0}
\end{block}

\column{0.4\linewidth}
\begin{block}{Unit}
No Unit
\end{block}

\begin{block}{Example}
\texttt{VDWDF\_GEN\_KERNEL}: \texttt{1}
\end{block}
\end{columns}

\begin{block}{Description}
Flag for computing the kernel functions, its 2nd derivatives and the 2nd derivative of spline functions. If 0 is set, the program will read the functions (vdWDF\_kernel.mat, vdWDF\_d2Phidk2.mat and vdWDF\_D2yDx2.mat in $\backslash$vdW$\backslash$vdWDF folder) directly; if 1 is set, the program will generate these functions before computation.
\end{block}
\end{frame}
%%%%%%%%%%%%%%%%%%%%%%%%%%%%%%%%%%%%%%%%%%%%%%%%%%%%%%%%%%%%%%%%%%%%%%%%%%%%%%%%%%%%%%%%%%%%%

%%%%%%%%%%%%%%%%%%%%%%%%%%%%%%%%%%%%%%%%%%%%%%%%%%%%%%%%%%%%%%%%%%%%%%%%%%%%%%%%%%%%%%%%%%%%%
\begin{frame}[allowframebreaks]{\texttt{EXX\_RANGE\_FOCK}} \label{EXX_RANGE_FOCK}
\vspace*{-12pt}
\begin{columns}
\column{0.35\linewidth}
\begin{block}{Type}
Double
\end{block}

\begin{block}{Default}
0.1587
\end{block}

\column{0.55\linewidth}
\begin{block}{Unit}
No unit
\end{block}

\begin{block}{Example}
\texttt{EXX\_RANGE\_FOCK}: \texttt{0.106}
\end{block}
\end{columns}

\begin{block}{Description}
Short range screen parameter of hartree-fock operator in HSE functional. 
\end{block}

\begin{block}{Remark}
Default is using VASP's value. Different code has different parameters. Be careful with the results. 
\end{block}

\end{frame}
%%%%%%%%%%%%%%%%%%%%%%%%%%%%%%%%%%%%%%%%%%%%%%%%%%%%%%%%%%%%%%%%%%%%%%%%%%%%%%%%%%%%%%%%%%%%%


%%%%%%%%%%%%%%%%%%%%%%%%%%%%%%%%%%%%%%%%%%%%%%%%%%%%%%%%%%%%%%%%%%%%%%%%%%%%%%%%%%%%%%%%%%%%%
\begin{frame}[allowframebreaks]{\texttt{EXX\_RANGE\_PBE}} \label{EXX_RANGE_PBE}
\vspace*{-12pt}
\begin{columns}
\column{0.35\linewidth}
\begin{block}{Type}
Double
\end{block}

\begin{block}{Default}
0.1587
\end{block}

\column{0.55\linewidth}
\begin{block}{Unit}
No unit
\end{block}

\begin{block}{Example}
\texttt{EXX\_RANGE\_PBE}: \texttt{0.106}
\end{block}
\end{columns}

\begin{block}{Description}
Short range screen parameter of PBE in HSE functional. 
\end{block}

\begin{block}{Remark}
Default is using VASP's value. Different code has different parameters. Be careful with the results. 
\end{block}

\end{frame}
%%%%%%%%%%%%%%%%%%%%%%%%%%%%%%%%%%%%%%%%%%%%%%%%%%%%%%%%%%%%%%%%%%%%%%%%%%%%%%%%%%%%%%%%%%%%%

%%%%%%%%%%%%%%%%%%%%%%%%%%%%%%%%%%%%%%%%%%%%%%%%%%%%%%%%%%%%%%%%%%%%%%%%%%%%%%%%%%%%%%%%%%%%%
\begin{frame}[allowframebreaks,c]{} \label{System:ion}

\begin{center}
\Huge \textbf{System: .ion file}
\end{center}

\end{frame}
%%%%%%%%%%%%%%%%%%%%%%%%%%%%%%%%%%%%%%%%%%%%%%%%%%%%%%%%%%%%%%%%%%%%%%%%%%%%%%%%%%%%%%%%%%%%%

%%%%%%%%%%%%%%%%%%%%%%%%%%%%%%%%%%%%%%%%%%%%%%%%%%%%%%%%%%%%%%%%%%%%%%%%%%%%%%%%%%%%%%%%%%%%%
\begin{frame}[allowframebreaks]{\texttt{ATOM\_TYPE}} \label{ATOM_TYPE}
\vspace*{-12pt}
\begin{columns}
\column{0.4\linewidth}
\begin{block}{Type}
String
\end{block}

\begin{block}{Default}
None
\end{block}

\column{0.4\linewidth}
\begin{block}{Unit}
No unit
\end{block}

\begin{block}{Example}
\texttt{ATOM\_TYPE: Fe}  
\end{block}
\end{columns}

\begin{block}{Description}
Atomic type symbol. 
\end{block}

\begin{block}{Remark}
The atomic type symbol can be attached with a number, e.g., Fe1 and Fe2. This feature is useful if one needs to provide two different pseudopotential files (\hyperlink{PSEUDO_POT}{\texttt{PSEUDO\_POT}}) for the same element.
\end{block}

\end{frame}
%%%%%%%%%%%%%%%%%%%%%%%%%%%%%%%%%%%%%%%%%%%%%%%%%%%%%%%%%%%%%%%%%%%%%%%%%%%%%%%%%%%%%%%%%%%%%


%%%%%%%%%%%%%%%%%%%%%%%%%%%%%%%%%%%%%%%%%%%%%%%%%%%%%%%%%%%%%%%%%%%%%%%%%%%%%%%%%%%%%%%%%%%%%
\begin{frame}[allowframebreaks]{\texttt{PSEUDO\_POT}} \label{PSEUDO_POT}
\vspace*{-12pt}
\begin{columns}
\column{0.4\linewidth}
\begin{block}{Type}
String
\end{block}

\begin{block}{Default}
None
\end{block}

\column{0.5\linewidth}
\begin{block}{Unit}
No unit
\end{block}

\begin{block}{Example}
\texttt{PSEUDO\_POT: ../psp/Fe.psp8}  
\end{block}
\end{columns}

\begin{block}{Description}
Path to the pseudopotential file. 
\end{block}

\begin{block}{Remark}
The default directory for the pseudopotential files is the same as the input files. For example, if a pseudopotential Fe.psp8 is put in the same directory as the input files, one can simply specify \texttt{PSEUDO\_POT: Fe.psp8}. 
\end{block}

\end{frame}
%%%%%%%%%%%%%%%%%%%%%%%%%%%%%%%%%%%%%%%%%%%%%%%%%%%%%%%%%%%%%%%%%%%%%%%%%%%%%%%%%%%%%%%%%%%%%


%%%%%%%%%%%%%%%%%%%%%%%%%%%%%%%%%%%%%%%%%%%%%%%%%%%%%%%%%%%%%%%%%%%%%%%%%%%%%%%%%%%%%%%%%%%%%
\begin{frame}[allowframebreaks]{\texttt{N\_TYPE\_ATOM}} \label{N_TYPE_ATOM}
\vspace*{-12pt}
\begin{columns}
\column{0.4\linewidth}
\begin{block}{Type}
Integer
\end{block}

\begin{block}{Default}
None
\end{block}

\column{0.5\linewidth}
\begin{block}{Unit}
No unit
\end{block}

\begin{block}{Example}
\texttt{N\_TYPE\_ATOM: 2}  
\end{block}
\end{columns}

\begin{block}{Description}
The number of atoms of a \hyperlink{ATOM_TYPE}{\texttt{ATOM\_TYPE}} specified immediately before this variable.
\end{block}

\begin{block}{Remark}
For a system with different types of atoms, one has to specify the number of atoms for every type.
\end{block}

\end{frame}
%%%%%%%%%%%%%%%%%%%%%%%%%%%%%%%%%%%%%%%%%%%%%%%%%%%%%%%%%%%%%%%%%%%%%%%%%%%%%%%%%%%%%%%%%%%%%



%%%%%%%%%%%%%%%%%%%%%%%%%%%%%%%%%%%%%%%%%%%%%%%%%%%%%%%%%%%%%%%%%%%%%%%%%%%%%%%%%%%%%%%%%%%%%
\begin{frame}[allowframebreaks]{\texttt{COORD}} \label{COORD}
\vspace*{-12pt}
\begin{columns}
\column{0.4\linewidth}
\begin{block}{Type}
Double
\end{block}

\begin{block}{Default}
None
\end{block}

\column{0.5\linewidth}
\begin{block}{Unit}
Bohr
\end{block}

\begin{block}{Example}
\texttt{COORD: \\}
0.0 0.0 0.0 \\
2.5 2.5 2.5
\end{block}
\end{columns}

\begin{block}{Description}
The Cartesian coordinates of atoms of a \hyperlink{ATOM_TYPE}{\texttt{ATOM\_TYPE}} specified before this variable. If the coordinates are outside the fundamental domain (see \hyperlink{CELL}{\texttt{CELL}} and \hyperlink{LATVEC}{\texttt{LATVEC}}) in the periodic directions (see \hyperlink{BC}{\texttt{BC}}), it will be automatically mapped back to the domain.
\end{block}

\begin{block}{Remark}
For a system with different types of atoms, one has to specify the coordinates for every \hyperlink{ATOM_TYPE}{\texttt{ATOM\_TYPE}}. One can also specify the coordinates of the atoms using \hyperlink{COORD_FRAC}{\texttt{COORD\_FRAC}}.
\end{block}

\end{frame}
%%%%%%%%%%%%%%%%%%%%%%%%%%%%%%%%%%%%%%%%%%%%%%%%%%%%%%%%%%%%%%%%%%%%%%%%%%%%%%%%%%%%%%%%%%%%%



%%%%%%%%%%%%%%%%%%%%%%%%%%%%%%%%%%%%%%%%%%%%%%%%%%%%%%%%%%%%%%%%%%%%%%%%%%%%%%%%%%%%%%%%%%%%%
\begin{frame}[allowframebreaks]{\texttt{COORD\_FRAC}} \label{COORD_FRAC}
\vspace*{-12pt}
\begin{columns}
\column{0.4\linewidth}
\begin{block}{Type}
Double
\end{block}

\begin{block}{Default}
None
\end{block}

\column{0.5\linewidth}
\begin{block}{Unit}
None
\end{block}

\begin{block}{Example}
\texttt{COORD\_FRAC: \\}
0.5 0.5 0.0 \\
0.0 0.5 0.5
\end{block}
\end{columns}

\begin{block}{Description}
The fractional coordinates of atoms of a \hyperlink{ATOM_TYPE}{\texttt{ATOM\_TYPE}} specified before this variable. \texttt{COORD\_FRAC}$(i,j)$ $\times$ \hyperlink{CELL}{\texttt{CELL}}$(j)$, $(j = 1,2,3)$ gives the coordinate of the $i^{th}$ atom along the $j^{th}$ \hyperlink{LATVEC}{\texttt{LATVEC}} direction. If the coordinates are outside the fundamental domain (see \hyperlink{CELL}{\texttt{CELL}} and \hyperlink{LATVEC}{\texttt{LATVEC}}) in the periodic directions (see \hyperlink{BC}{\texttt{BC}}), it will be automatically mapped back to the domain.
\end{block}

\begin{block}{Remark}
For a system with different types of atoms, one has to specify the coordinates for every \hyperlink{ATOM_TYPE}{\texttt{ATOM\_TYPE}}. One can also specify the coordinates of the atoms using \hyperlink{COORD}{\texttt{COORD}}.
\end{block}

\end{frame}
%%%%%%%%%%%%%%%%%%%%%%%%%%%%%%%%%%%%%%%%%%%%%%%%%%%%%%%%%%%%%%%%%%%%%%%%%%%%%%%%%%%%%%%%%%%%%



%%%%%%%%%%%%%%%%%%%%%%%%%%%%%%%%%%%%%%%%%%%%%%%%%%%%%%%%%%%%%%%%%%%%%%%%%%%%%%%%%%%%%%%%%%%%%
\begin{frame}[allowframebreaks]{\texttt{RELAX}} \label{RELAX}
\vspace*{-12pt}
\begin{columns}
\column{0.4\linewidth}
\begin{block}{Type}
Integer
\end{block}

\begin{block}{Default}
1 1 1
\end{block}

\column{0.5\linewidth}
\begin{block}{Unit}
No unit
\end{block}

\begin{block}{Example}
\texttt{RELAX: \\}
1 0 1 \\
0 1 0
\end{block}
\end{columns}

\begin{block}{Description}
Atomic coordinate with the corresponding \texttt{RELAX} value 0 is held fixed during relaxation/MD.
\end{block}

\end{frame}
%%%%%%%%%%%%%%%%%%%%%%%%%%%%%%%%%%%%%%%%%%%%%%%%%%%%%%%%%%%%%%%%%%%%%%%%%%%%%%%%%%%%%%%%%%%%%


%%%%%%%%%%%%%%%%%%%%%%%%%%%%%%%%%%%%%%%%%%%%%%%%%%%%%%%%%%%%%%%%%%%%%%%%%%%%%%%%%%%%%%%%%%%%%
\begin{frame}[allowframebreaks]{\texttt{SPIN}} \label{SPIN}
\vspace*{-12pt}
\begin{columns}
\column{0.4\linewidth}
\begin{block}{Type}
Double
\end{block}

\begin{block}{Default}
0.0
\end{block}

\column{0.5\linewidth}
\begin{block}{Unit}
No unit
\end{block}

\begin{block}{Example}
\texttt{SPIN: \\}
0 0 1.0 \\
0 0 -1.0
\end{block}
\end{columns}

\begin{block}{Description}
Specifies the net initial spin on each atom for a spin-polarized calculation. If collinear spin used, user could use either 1 column of data for z-direction of each atom, or 3 columns of data with 0 on the first 2 columns (x,y-directions). For noncollinear spin, use need to use 3 columns of data for all directions. 
\end{block}

\end{frame}
%%%%%%%%%%%%%%%%%%%%%%%%%%%%%%%%%%%%%%%%%%%%%%%%%%%%%%%%%%%%%%%%%%%%%%%%%%%%%%%%%%%%%%%%%%%%%

%%%%%%%%%%%%%%%%%%%%%%%%%%%%%%%%%%%%%%%%%%%%%%%%%%%%%%%%%%%%%%%%%%%%%%%%%%%%%%%%%%%%%%%%%%%%%
\begin{frame}[allowframebreaks]{\texttt{HUBBARD}} \label{HUBBARD}
\vspace*{-12pt}
\begin{columns}
\column{0.4\linewidth}
\begin{block}{Type}
None
\end{block}

\begin{block}{Default}
None
\end{block}

\column{0.5\linewidth}
\begin{block}{Unit}
No unit
\end{block}

\begin{block}{Example}
\texttt{HUBBARD: \\}
\texttt{U\_ATOM\_TYPE: Ni \\}
\texttt{U\_VAL:} 0 0 0.05 0 \\
\end{block}
\end{columns}

\begin{block}{Description}
Triggers a DFT+U calculation on top of the \hyperlink{EXCHANGE_CORRELATION}{\texttt{EXCHANGE\_CORRELATION}} specified in the \texttt{.inpt} file. Must be followed by specifying the atoms on which a U correction is desired along with value of U as per Dudarev's scheme.
\end{block}

\end{frame}
%%%%%%%%%%%%%%%%%%%%%%%%%%%%%%%%%%%%%%%%%%%%%%%%%%%%%%%%%%%%%%%%%%%%%%%%%%%%%%%%%%%%%%%%%%%%%

%%%%%%%%%%%%%%%%%%%%%%%%%%%%%%%%%%%%%%%%%%%%%%%%%%%%%%%%%%%%%%%%%%%%%%%%%%%%%%%%%%%%%%%%%%%%%
\begin{frame}[allowframebreaks]{\texttt{U\_ATOM\_TYPE}} \label{U_ATOM_TYPE}
\vspace*{-12pt}
\begin{columns}
\column{0.4\linewidth}
\begin{block}{Type}
String
\end{block}

\begin{block}{Default}
None
\end{block}

\column{0.5\linewidth}
\begin{block}{Unit}
No unit
\end{block}

\begin{block}{Example}
\texttt{U\_ATOM\_TYPE: Ni \\}
\end{block}
\end{columns}

\begin{block}{Description}
Atomic type symbol for which a U correction is desired. Must be specified within the \hyperlink{HUBBARD}{\texttt{HUBBARD}} block. The atomic type should already have been specified above in the \texttt{.ion} file.
\end{block}

\end{frame}
%%%%%%%%%%%%%%%%%%%%%%%%%%%%%%%%%%%%%%%%%%%%%%%%%%%%%%%%%%%%%%%%%%%%%%%%%%%%%%%%%%%%%%%%%%%%%

%%%%%%%%%%%%%%%%%%%%%%%%%%%%%%%%%%%%%%%%%%%%%%%%%%%%%%%%%%%%%%%%%%%%%%%%%%%%%%%%%%%%%%%%%%%%%
\begin{frame}[allowframebreaks]{\texttt{U\_VAL}} \label{U_VAL}
\vspace*{-12pt}
\begin{columns}
\column{0.4\linewidth}
\begin{block}{Type}
Double
\end{block}

\begin{block}{Default}
None
\end{block}

\column{0.5\linewidth}
\begin{block}{Unit}
Hartree
\end{block}

\begin{block}{Example}
\texttt{U\_VAL:} 0 0 0.1 0 \\
\end{block}
\end{columns}

\begin{block}{Description}
Value of effective \texttt{U} for every azimuthal quantum number (s p d f orbitals) of the atom on which the correction is applied per Dudarev's scheme. By default, the corrections will only be applied to the outermost states available in the corresponding \hyperlink{PSEUDO_POT}{\texttt{PSEUDO\_POT}}. Must be specified after specifying the \hyperlink{U_ATOM_TYPE}{\texttt{U\_ATOM\_TYPE}}.
\end{block}

\end{frame}
%%%%%%%%%%%%%%%%%%%%%%%%%%%%%%%%%%%%%%%%%%%%%%%%%%%%%%%%%%%%%%%%%%%%%%%%%%%%%%%%%%%%%%%%%%%%%

%%%%%%%%%%%%%%%%%%%%%%%%%%%%%%%%%%%%%%%%%%%%%%%%%%%%%%%%%%%%%%%%%%%%%%%%%%%%%%%%%%%%%%%%%%%%%
\begin{frame}[allowframebreaks]{\texttt{CORE\_FLAG}} \label{CORE_FLAG}
\vspace*{-12pt}
\begin{columns}
\column{0.4\linewidth}
\begin{block}{Type}
Integer
\end{block}

\begin{block}{Default}
0
\end{block}

\column{0.5\linewidth}
\begin{block}{Unit}
No unit
\end{block}

\begin{block}{Example}
\texttt{CORE\_FLAG:} 1 \\
\end{block}
\end{columns}

\begin{block}{Description}
Must be set to 1 if corrections are desired on all the inner states for each s/p/d/f orbitals available in the corresponding \hyperlink{PSEUDO_POT}{\texttt{PSEUDO\_POT}}. Must be specified after specifying the \hyperlink{U_ATOM_TYPE}{\texttt{U\_ATOM\_TYPE}}.
\end{block}

\end{frame}
%%%%%%%%%%%%%%%%%%%%%%%%%%%%%%%%%%%%%%%%%%%%%%%%%%%%%%%%%%%%%%%%%%%%%%%%%%%%%%%%%%%%%%%%%%%%%


%%%%%%%%%%%%%%%%%%%%%%%%%%%%%%%%%%%%%%%%%%%%%%%%%%%%%%%%%%%%%%%%%%%%%%%%%%%%%%%%%%%%%%%%%%%%%
\begin{frame}[allowframebreaks,c]{} \label{SCF}

\begin{center}
\Huge \textbf{SCF}
\end{center}

\end{frame}
%%%%%%%%%%%%%%%%%%%%%%%%%%%%%%%%%%%%%%%%%%%%%%%%%%%%%%%%%%%%%%%%%%%%%%%%%%%%%%%%%%%%%%%%%%%%%


%%%%%%%%%%%%%%%%%%%%%%%%%%%%%%%%%%%%%%%%%%%%%%%%%%%%%%%%%%%%%%%%%%%%%%%%%%%%%%%%%%%%%%%%%%%%%
\begin{frame}[allowframebreaks]{\texttt{CHEB\_DEGREE}} \label{CHEB_DEGREE}
\vspace*{-12pt}
\begin{columns}
\column{0.4\linewidth}
\begin{block}{Type}
Integer
\end{block}

\begin{block}{Default}
Automatically set.
\end{block}

\column{0.4\linewidth}
\begin{block}{Unit}
No unit
\end{block}

\begin{block}{Example}
\texttt{CHEB\_DEGEE}: 25
\end{block}
\end{columns}

\begin{block}{Description}
Degree of polynomial used for Chebyshev filtering. 
\end{block}

\begin{block}{Remark}
For larger mesh-sizes, smaller values of \texttt{CHEB\_DEGREE} are generally more efficient, and vice-versa.
\end{block}

\end{frame}
%%%%%%%%%%%%%%%%%%%%%%%%%%%%%%%%%%%%%%%%%%%%%%%%%%%%%%%%%%%%%%%%%%%%%%%%%%%%%%%%%%%%%%%%%%%%%


%%%%%%%%%%%%%%%%%%%%%%%%%%%%%%%%%%%%%%%%%%%%%%%%%%%%%%%%%%%%%%%%%%%%%%%%%%%%%%%%%%%%%%%%%%%%%
\begin{frame}[allowframebreaks]{\texttt{RHO\_TRIGGER}} \label{RHO_TRIGGER}
\vspace*{-12pt}
\begin{columns}
\column{0.4\linewidth}
\begin{block}{Type}
Integer
\end{block}

\begin{block}{Default}
4 or 6
\end{block}

\column{0.4\linewidth}
\begin{block}{Unit}
No unit
\end{block}

\begin{block}{Example}
\texttt{RHO\_TRIGGER}: 5
\end{block}
\end{columns}

\begin{block}{Description}
The number of times Chebyshev filtering is repeated before updating the electron density in the very first SCF iteration.
\end{block}

\begin{block}{Remark}
Values smaller than the default value of 4 can result in a significant increase in the number of SCF
iterations. Larger values can sometimes reduce the number of SCF iterations. For non-collinear spin calculation, default is 6 otherwise 4.
\end{block}

\end{frame}
%%%%%%%%%%%%%%%%%%%%%%%%%%%%%%%%%%%%%%%%%%%%%%%%%%%%%%%%%%%%%%%%%%%%%%%%%%%%%%%%%%%%%%%%%%%%%


%%%%%%%%%%%%%%%%%%%%%%%%%%%%%%%%%%%%%%%%%%%%%%%%%%%%%%%%%%%%%%%%%%%%%%%%%%%%%%%%%%%%%%%%%%%%%
\begin{frame}[allowframebreaks]{\texttt{NUM\_CHEFSI}} \label{NUM_CHEFSI}
\vspace*{-12pt}
\begin{columns}
\column{0.4\linewidth}
\begin{block}{Type}
Integer
\end{block}

\begin{block}{Default}
1
\end{block}

\column{0.4\linewidth}
\begin{block}{Unit}
No unit
\end{block}

\begin{block}{Example}
\texttt{NUM\_CHEFSI}: 2
\end{block}
\end{columns}

\begin{block}{Description}
The number of times ChefSI algorithm is repeated in SCF iteration except the first one, which is controlled by \texttt{RHO\_TRIGGER}.
\end{block}

\begin{block}{Remark}
For non-collinear spin calculation, it might helped SCF convergence in some cases. 
\end{block}

\end{frame}
%%%%%%%%%%%%%%%%%%%%%%%%%%%%%%%%%%%%%%%%%%%%%%%%%%%%%%%%%%%%%%%%%%%%%%%%%%%%%%%%%%%%%%%%%%%%%


%%%%%%%%%%%%%%%%%%%%%%%%%%%%%%%%%%%%%%%%%%%%%%%%%%%%%%%%%%%%%%%%%%%%%%%%%%%%%%%%%%%%%%%%%%%%%
\begin{frame}[allowframebreaks]{\texttt{{MAXIT\_SCF}}} \label{MAXIT_SCF}
\vspace*{-12pt}
\begin{columns}
\column{0.4\linewidth}
\begin{block}{Type}
Integer
\end{block}

\begin{block}{Default}
100
\end{block}

\column{0.4\linewidth}
\begin{block}{Unit}
No unit
\end{block}

\begin{block}{Example}
\texttt{MAXIT\_SCF}: 50
\end{block}
\end{columns}

\begin{block}{Description}
Maximum number of SCF iterations.
\end{block}

\begin{block}{Remark}
Larger values than the default of 100 may be required for highly inhomogeneous systems, particularly when small values of \hyperlink{SMEARING}{\texttt{SMEARING}}/\hyperlink{ELEC_TEMP}{\texttt{ELEC\_TEMP}} are chosen. 
\end{block}

\end{frame}
%%%%%%%%%%%%%%%%%%%%%%%%%%%%%%%%%%%%%%%%%%%%%%%%%%%%%%%%%%%%%%%%%%%%%%%%%%%%%%%%%%%%%%%%%%%%%



%%%%%%%%%%%%%%%%%%%%%%%%%%%%%%%%%%%%%%%%%%%%%%%%%%%%%%%%%%%%%%%%%%%%%%%%%%%%%%%%%%%%%%%%%%%%%
\begin{frame}[allowframebreaks]{\texttt{TOL\_SCF}} \label{TOL_SCF}
\vspace*{-12pt}
\begin{columns}
\column{0.4\linewidth}
\begin{block}{Type}
Double
\end{block}

\begin{block}{Default}
see description
\end{block}

\column{0.4\linewidth}
\begin{block}{Unit}
No unit
\end{block}

\begin{block}{Example}
\texttt{TOL\_SCF}: 1e-5
\end{block}
\end{columns}

\begin{block}{Description}
In case of single point calculation, \texttt{TOL\_SCF} is set for $10^{-5}$ Ha/atom energy accuracy.
In case of MD, \texttt{TOL\_SCF} is set for $10^{-3}$ Ha/Bohr force accuracy.
In case of relaxation, \texttt{TOL\_SCF} is set for \hyperlink{TOL_RELAX}{\texttt{TOL\_RELAX}}/5 Ha/Bohr force accuracy. \\
The tolerance on the normalized residual of the effective potential or the electron density for convergence of the SCF iteration. 
\end{block}

\begin{block}{Remark}
Only one of \hyperlink{TOL_SCF}{\texttt{TOL\_SCF}}, \hyperlink{SCF_ENERGY_ACC}{\texttt{SCF\_ENERGY\_ACC}}, or \hyperlink{SCF_FORCE_ACC}{\texttt{SCF\_FORCE\_ACC}} can be specified.
\end{block}

\end{frame}
%%%%%%%%%%%%%%%%%%%%%%%%%%%%%%%%%%%%%%%%%%%%%%%%%%%%%%%%%%%%%%%%%%%%%%%%%%%%%%%%%%%%%%%%%%%%%




%%%%%%%%%%%%%%%%%%%%%%%%%%%%%%%%%%%%%%%%%%%%%%%%%%%%%%%%%%%%%%%%%%%%%%%%%%%%%%%%%%%%%%%%%%%%%
\begin{frame}[allowframebreaks]{\texttt{SCF\_FORCE\_ACC}} \label{SCF_FORCE_ACC}
\vspace*{-12pt}
\begin{columns}
\column{0.4\linewidth}
\begin{block}{Type}
Double
\end{block}

\begin{block}{Default}
None
\end{block}

\column{0.4\linewidth}
\begin{block}{Unit}
Ha/Bohr
\end{block}

\begin{block}{Example}
\texttt{SCF\_FORCE\_ACC}: 1e-4
\end{block}
\end{columns}

\begin{block}{Description}
The tolerance on the atomic forces for convergence of the SCF iteration. 
\end{block}

\begin{block}{Remark}
Only one of \hyperlink{SCF_FORCE_ACC}{\texttt{SCF\_FORCE\_ACC}}, \hyperlink{TOL_SCF}{\texttt{TOL\_SCF}} or \hyperlink{SCF_ENERGY_ACC}{\texttt{SCF\_ENERGY\_ACC}} can be specified.
\end{block}

\end{frame}
%%%%%%%%%%%%%%%%%%%%%%%%%%%%%%%%%%%%%%%%%%%%%%%%%%%%%%%%%%%%%%%%%%%%%%%%%%%%%%%%%%%%%%%%%%%%%


%%%%%%%%%%%%%%%%%%%%%%%%%%%%%%%%%%%%%%%%%%%%%%%%%%%%%%%%%%%%%%%%%%%%%%%%%%%%%%%%%%%%%%%%%%%%%
\begin{frame}[allowframebreaks]{\texttt{SCF\_ENERGY\_ACC}} \label{SCF_ENERGY_ACC}
\vspace*{-12pt}
\begin{columns}
\column{0.4\linewidth}
\begin{block}{Type}
Double
\end{block}

\begin{block}{Default}
None
\end{block}

\column{0.4\linewidth}
\begin{block}{Unit}
Ha/atom
\end{block}

\begin{block}{Example}
\texttt{SCF\_ENERGY\_ACC}: 1e-5
\end{block}
\end{columns}

\begin{block}{Description}
The tolerance on the free energy for the convergence of the SCF iteration. 
\end{block}

\begin{block}{Remark}
Only one of \hyperlink{SCF_ENERGY_ACC}{\texttt{SCF\_ENERGY\_ACC}}, \hyperlink{SCF_FORCE_ACC}{\texttt{SCF\_FORCE\_ACC}}, or \hyperlink{TOL_SCF}{\texttt{TOL\_SCF}} can be specified.
\end{block}

\end{frame}
%%%%%%%%%%%%%%%%%%%%%%%%%%%%%%%%%%%%%%%%%%%%%%%%%%%%%%%%%%%%%%%%%%%%%%%%%%%%%%%%%%%%%%%%%%%%%




%%%%%%%%%%%%%%%%%%%%%%%%%%%%%%%%%%%%%%%%%%%%%%%%%%%%%%%%%%%%%%%%%%%%%%%%%%%%%%%%%%%%%%%%%%%%%
\begin{frame}[allowframebreaks]{\texttt{TOL\_LANCZOS}} \label{TOL_LANCZOS}
\vspace*{-12pt}
\begin{columns}
\column{0.4\linewidth}
\begin{block}{Type}
Double
\end{block}

\begin{block}{Default}
1e-2
\end{block}

\column{0.4\linewidth}
\begin{block}{Unit}
No unit
\end{block}

\begin{block}{Example}
\texttt{TOL\_LANCZOS}: 1e-3
\end{block}
\end{columns}

\begin{block}{Description}
The tolerance within the Lanczos algorithm for calculating the extremal eigenvalues of the Hamiltonian, required as part of the CheFSI method. 
\end{block}

\begin{block}{Remark}
Typically, the Lanczos tolerance does not need to be very strict.
\end{block}

\end{frame}
%%%%%%%%%%%%%%%%%%%%%%%%%%%%%%%%%%%%%%%%%%%%%%%%%%%%%%%%%%%%%%%%%%%%%%%%%%%%%%%%%%%%%%%%%%%%%




%%%%%%%%%%%%%%%%%%%%%%%%%%%%%%%%%%%%%%%%%%%%%%%%%%%%%%%%%%%%%%%%%%%%%%%%%%%%%%%%%%%%%%%%%%%%%
\begin{frame}[allowframebreaks]{\texttt{MIXING\_VARIABLE}} \label{MIXING_VARIABLE}
\vspace*{-12pt}
\begin{columns}
\column{0.4\linewidth}
\begin{block}{Type}
String
\end{block}

\begin{block}{Default}
\texttt{potential}
\end{block}

\column{0.4\linewidth}
\begin{block}{Unit}
No unit
\end{block}

\begin{block}{Example}
\texttt{MIXING\_VARIABLE}: \texttt{density}
\end{block}
\end{columns}

\begin{block}{Description}
This specifies whether potential or density mixing is performed in the SCF iteration. Available options are: \texttt{potential} and \texttt{density}.
\end{block}

\end{frame}
%%%%%%%%%%%%%%%%%%%%%%%%%%%%%%%%%%%%%%%%%%%%%%%%%%%%%%%%%%%%%%%%%%%%%%%%%%%%%%%%%%%%%%%%%%%%%



%%%%%%%%%%%%%%%%%%%%%%%%%%%%%%%%%%%%%%%%%%%%%%%%%%%%%%%%%%%%%%%%%%%%%%%%%%%%%%%%%%%%%%%%%%%%%
\begin{frame}[allowframebreaks]{\texttt{MIXING\_HISTORY}} \label{MIXING_HISTORY}
\vspace*{-12pt}
\begin{columns}
\column{0.4\linewidth}
\begin{block}{Type}
Integer
\end{block}

\begin{block}{Default}
7
\end{block}

\column{0.4\linewidth}
\begin{block}{Unit}
No unit
\end{block}

\begin{block}{Example}
\texttt{MIXING\_HISTORY}: 40
\end{block}
\end{columns}

\begin{block}{Description}
The mixing history used in Pulay mixing.
\end{block}

\begin{block}{Remark}
Too small values of \hyperlink{MIXING_HISTORY}{\texttt{MIXING\_HISTORY}} can result in poor SCF convergence.
\end{block}

\end{frame}
%%%%%%%%%%%%%%%%%%%%%%%%%%%%%%%%%%%%%%%%%%%%%%%%%%%%%%%%%%%%%%%%%%%%%%%%%%%%%%%%%%%%%%%%%%%%%



%%%%%%%%%%%%%%%%%%%%%%%%%%%%%%%%%%%%%%%%%%%%%%%%%%%%%%%%%%%%%%%%%%%%%%%%%%%%%%%%%%%%%%%%%%%%%
\begin{frame}[allowframebreaks]{\texttt{MIXING\_PARAMETER}} \label{MIXING_PARAMETER}
\vspace*{-12pt}
\begin{columns}
\column{0.4\linewidth}
\begin{block}{Type}
Double
\end{block}

\begin{block}{Default}
0.3
\end{block}

\column{0.4\linewidth}
\begin{block}{Unit}
No unit
\end{block}

\begin{block}{Example}
\texttt{MIXING\_PARAMETER}: 0.1
\end{block}
\end{columns}

\begin{block}{Description}
The value of the relaxation parameter used in Pulay/simple mixing.
\end{block}

\begin{block}{Remark}
Values larger than the default value of 0.3 can be used for insulating systems, whereas smaller values are generally required for metallic  systems, particularly at small values of \hyperlink{SMEARING}{\texttt{SMEARING}} or \hyperlink{ELEC_TEMP}{\texttt{ELEC\_TEMP}}.
\end{block}

\end{frame}
%%%%%%%%%%%%%%%%%%%%%%%%%%%%%%%%%%%%%%%%%%%%%%%%%%%%%%%%%%%%%%%%%%%%%%%%%%%%%%%%%%%%%%%%%%%%%


%%%%%%%%%%%%%%%%%%%%%%%%%%%%%%%%%%%%%%%%%%%%%%%%%%%%%%%%%%%%%%%%%%%%%%%%%%%%%%%%%%%%%%%%%%%%%
\begin{frame}[allowframebreaks]{\texttt{MIXING\_PARAMETER\_SIMPLE}} \label{MIXING_PARAMETER_SIMPLE}
\vspace*{-12pt}
\begin{columns}
\column{0.4\linewidth}
\begin{block}{Type}
Double
\end{block}

\begin{block}{Default}
Automatically set to the same as \hyperlink{MIXING_PARAMETER}{\texttt{MIXING\_PARAMETER}}
\end{block}

\column{0.4\linewidth}
\begin{block}{Unit}
No unit
\end{block}

\begin{block}{Example}
\texttt{MIXING\_PARAMETER\_SIMPLE}: 0.1
\end{block}
\end{columns}

\begin{block}{Description}
The value of the relaxation parameter used in the simple mixing step in the periodic Pulay scheme.
\end{block}

\end{frame}
%%%%%%%%%%%%%%%%%%%%%%%%%%%%%%%%%%%%%%%%%%%%%%%%%%%%%%%%%%%%%%%%%%%%%%%%%%%%%%%%%%%%%%%%%%%%%


%%%%%%%%%%%%%%%%%%%%%%%%%%%%%%%%%%%%%%%%%%%%%%%%%%%%%%%%%%%%%%%%%%%%%%%%%%%%%%%%%%%%%%%%%%%%%
\begin{frame}[allowframebreaks]{\texttt{MIXING\_PARAMETER\_MAG}} \label{MIXING_PARAMETER_MAG}
\vspace*{-12pt}
\begin{columns}
\column{0.4\linewidth}
\begin{block}{Type}
Double
\end{block}

\begin{block}{Default}
Automatically set to the same as \hyperlink{MIXING_PARAMETER}{\texttt{MIXING\_PARAMETER}}.
\end{block}

\column{0.4\linewidth}
\begin{block}{Unit}
No unit
\end{block}

\begin{block}{Example}
\texttt{MIXING\_PARAMETER\_MAG}: 4.0
\end{block}
\end{columns}

\begin{block}{Description}
The mixing parameter for the magnetization density in Pulay mixing for spin-polarized calculations.
\end{block}

\begin{block}{Remark}
    For spin-polarized calculations, when SCF has difficulty to converge, increasing the mixing parameter to magnetization density might help. For example, setting it to 4.0, while turning off the preconditioner applied to the magnetization density (by setting \hyperlink{MIXING_PRECOND_MAG}{\texttt{MIXING\_PRECOND\_MAG}} to `\texttt{none}') is a good choice.
\end{block}

\end{frame}
%%%%%%%%%%%%%%%%%%%%%%%%%%%%%%%%%%%%%%%%%%%%%%%%%%%%%%%%%%%%%%%%%%%%%%%%%%%%%%%%%%%%%%%%%%%%%


%%%%%%%%%%%%%%%%%%%%%%%%%%%%%%%%%%%%%%%%%%%%%%%%%%%%%%%%%%%%%%%%%%%%%%%%%%%%%%%%%%%%%%%%%%%%%
\begin{frame}[allowframebreaks]{\texttt{MIXING\_PARAMETER\_SIMPLE\_MAG}} \label{MIXING_PARAMETER_SIMPLE_MAG}
\vspace*{-12pt}
\begin{columns}
\column{0.4\linewidth}
\begin{block}{Type}
Double
\end{block}

\begin{block}{Default}
Automatically set to the same as \hyperlink{MIXING_PARAMETER_MAG}{\texttt{MIXING\_PARAMETER\_MAG}}
\end{block}

\column{0.5\linewidth}
\begin{block}{Unit}
No unit
\end{block}

\begin{block}{Example}
\texttt{MIXING\_PARAMETER\_SIMPLE\_MAG}: 4.0
\end{block}
\end{columns}

\begin{block}{Description}
The value of the relaxation parameter for the magnetization density used in the simple mixing step in the periodic Pulay scheme for spin-polarized calculations.
\end{block}

\end{frame}
%%%%%%%%%%%%%%%%%%%%%%%%%%%%%%%%%%%%%%%%%%%%%%%%%%%%%%%%%%%%%%%%%%%%%%%%%%%%%%%%%%%%%%%%%%%%%



%%%%%%%%%%%%%%%%%%%%%%%%%%%%%%%%%%%%%%%%%%%%%%%%%%%%%%%%%%%%%%%%%%%%%%%%%%%%%%%%%%%%%%%%%%%%%
\begin{frame}[allowframebreaks]{\texttt{PULAY\_FREQUENCY}} \label{PULAY_FREQUENCY}
\vspace*{-12pt}
\begin{columns}
\column{0.4\linewidth}
\begin{block}{Type}
Integer
\end{block}

\begin{block}{Default}
1
\end{block}

\column{0.4\linewidth}
\begin{block}{Unit}
No unit
\end{block}

\begin{block}{Example}
\texttt{PULAY\_FREQUENCY}: 4
\end{block}
\end{columns}

\begin{block}{Description}
The frequency of Pulay mixing in Periodic Pulay. 
\end{block}

\begin{block}{Remark}
The default value of 1 corresponds to Pulay mixing.
\end{block}

\end{frame}
%%%%%%%%%%%%%%%%%%%%%%%%%%%%%%%%%%%%%%%%%%%%%%%%%%%%%%%%%%%%%%%%%%%%%%%%%%%%%%%%%%%%%%%%%%%%%




%%%%%%%%%%%%%%%%%%%%%%%%%%%%%%%%%%%%%%%%%%%%%%%%%%%%%%%%%%%%%%%%%%%%%%%%%%%%%%%%%%%%%%%%%%%%%
\begin{frame}[allowframebreaks]{\texttt{PULAY\_RESTART}} \label{PULAY_RESTART}
\vspace*{-12pt}
\begin{columns}
\column{0.4\linewidth}
\begin{block}{Type}
Integer
\end{block}

\begin{block}{Default}
0
\end{block}

\column{0.5\linewidth}
\begin{block}{Unit}
No unit
\end{block}

\begin{block}{Example}
\texttt{PULAY\_RESTART}: 1
\end{block}
\end{columns}

\begin{block}{Description}
The flag for restarting the `Periodic Pulay' mixing. If set to 0, the restarted Pulay method is turned off.
\end{block}

\end{frame}
%%%%%%%%%%%%%%%%%%%%%%%%%%%%%%%%%%%%%%%%%%%%%%%%%%%%%%%%%%%%%%%%%%%%%%%%%%%%%%%%%%%%%%%%%%%%%



%%%%%%%%%%%%%%%%%%%%%%%%%%%%%%%%%%%%%%%%%%%%%%%%%%%%%%%%%%%%%%%%%%%%%%%%%%%%%%%%%%%%%%%%%%%%%
\begin{frame}[allowframebreaks]{\texttt{MIXING\_PRECOND}} \label{MIXING_PRECOND}
\vspace*{-12pt}
\begin{columns}
\column{0.4\linewidth}
\begin{block}{Type}
String
\end{block}

\begin{block}{Default}
none
\end{block}

\column{0.4\linewidth}
\begin{block}{Unit}
No unit
\end{block}

\begin{block}{Example}
\texttt{MIXING\_PRECOND}: \texttt{kerker}
\end{block}
\end{columns}

\begin{block}{Description}
This specifies the preconditioner used in the SCF iteration. Available options are: \texttt{none}, \texttt{kerker}, \texttt{resta} and \texttt{truncated\_kerker}.
\end{block}

\end{frame}
%%%%%%%%%%%%%%%%%%%%%%%%%%%%%%%%%%%%%%%%%%%%%%%%%%%%%%%%%%%%%%%%%%%%%%%%%%%%%%%%%%%%%%%%%%%%%


%%%%%%%%%%%%%%%%%%%%%%%%%%%%%%%%%%%%%%%%%%%%%%%%%%%%%%%%%%%%%%%%%%%%%%%%%%%%%%%%%%%%%%%%%%%%%
\begin{frame}[allowframebreaks]{\texttt{MIXING\_PRECOND\_MAG}} \label{MIXING_PRECOND_MAG}
\vspace*{-12pt}
\begin{columns}
\column{0.4\linewidth}
\begin{block}{Type}
String
\end{block}

\begin{block}{Default}
\texttt{none}
\end{block}

\column{0.5\linewidth}
\begin{block}{Unit}
No unit
\end{block}

\begin{block}{Example}
\texttt{MIXING\_PRECOND\_MAG}: \texttt{kerker}
\end{block}
\end{columns}

\begin{block}{Description}
This specifies the preconditioner used for the magnetization density in the SCF iteration for spin-polarized calculations. Available options are: \texttt{none}, \texttt{kerker}.
\end{block}

\end{frame}
%%%%%%%%%%%%%%%%%%%%%%%%%%%%%%%%%%%%%%%%%%%%%%%%%%%%%%%%%%%%%%%%%%%%%%%%%%%%%%%%%%%%%%%%%%%%%



%%%%%%%%%%%%%%%%%%%%%%%%%%%%%%%%%%%%%%%%%%%%%%%%%%%%%%%%%%%%%%%%%%%%%%%%%%%%%%%%%%%%%%%%%%%%%
\begin{frame}[allowframebreaks]{\texttt{TOL\_PRECOND}} \label{TOL_PRECOND}
\vspace*{-12pt}
\begin{columns}
\column{0.4\linewidth}
\begin{block}{Type}
Double
\end{block}

\begin{block}{Default}
$h^2\times0.001$
\end{block}

\column{0.4\linewidth}
\begin{block}{Unit}
No unit
\end{block}

\begin{block}{Example}
\texttt{TOL\_PRECOND}: 1e-4
\end{block}
\end{columns}

\begin{block}{Description}
The tolerance on the relative residual for the linear systems arising during the real-space preconditioning of the SCF.
\end{block}

\begin{block}{Remark}
The linear systems do not need to be solved very accurately. $h$ is the mesh spacing.
\end{block}

\end{frame}
%%%%%%%%%%%%%%%%%%%%%%%%%%%%%%%%%%%%%%%%%%%%%%%%%%%%%%%%%%%%%%%%%%%%%%%%%%%%%%%%%%%%%%%%%%%%%



%%%%%%%%%%%%%%%%%%%%%%%%%%%%%%%%%%%%%%%%%%%%%%%%%%%%%%%%%%%%%%%%%%%%%%%%%%%%%%%%%%%%%%%%%%%%%
\begin{frame}[allowframebreaks]{\texttt{PRECOND\_KERKER\_KTF}} \label{PRECOND_KERKER_KTF}
\vspace*{-12pt}
\begin{columns}
\column{0.4\linewidth}
\begin{block}{Type}
Double
\end{block}

\begin{block}{Default}
1.0
\end{block}

\column{0.5\linewidth}
\begin{block}{Unit}
$\textrm{Bohr}^{-1}$
\end{block}

\begin{block}{Example}
\texttt{PRECOND\_KERKER\_KTF}: 0.8
\end{block}
\end{columns}

\begin{block}{Description}
The Thomas-Fermi screening length appearing in the \texttt{kerker} and \texttt{truncated\_kerker} preconditioners (\hyperlink{MIXING_PRECOND}{\texttt{MIXING\_PRECOND}}). 
\end{block}

\end{frame}
%%%%%%%%%%%%%%%%%%%%%%%%%%%%%%%%%%%%%%%%%%%%%%%%%%%%%%%%%%%%%%%%%%%%%%%%%%%%%%%%%%%%%%%%%%%%%



%%%%%%%%%%%%%%%%%%%%%%%%%%%%%%%%%%%%%%%%%%%%%%%%%%%%%%%%%%%%%%%%%%%%%%%%%%%%%%%%%%%%%%%%%%%%%
\begin{frame}[allowframebreaks]{\texttt{PRECOND\_KERKER\_THRESH}} \label{PRECOND_KERKER_THRESH}
\vspace*{-12pt}
\begin{columns}
\column{0.4\linewidth}
\begin{block}{Type}
Double
\end{block}

\begin{block}{Default}
0.25
\end{block}

\column{0.5\linewidth}
\begin{block}{Unit}
No unit
\end{block}

\begin{block}{Example}
\texttt{PRECOND\_KERKER\_THRESH}: 0.1
\end{block}
\end{columns}

\begin{block}{Description}
The threshold for the \texttt{truncated\_kerker} preconditioner (\hyperlink{MIXING_PRECOND}{\texttt{MIXING\_PRECOND}}).
\end{block}

\begin{block}{Remark}
This threshold will be scaled by the \hyperlink{MIXING_PARAMETER}{\texttt{MIXING\_PARAMETER}}. If the threshold is set to 0, the \texttt{kerker} preconditioner is recovered.
\end{block}

\end{frame}
%%%%%%%%%%%%%%%%%%%%%%%%%%%%%%%%%%%%%%%%%%%%%%%%%%%%%%%%%%%%%%%%%%%%%%%%%%%%%%%%%%%%%%%%%%%%%

%%%%%%%%%%%%%%%%%%%%%%%%%%%%%%%%%%%%%%%%%%%%%%%%%%%%%%%%%%%%%%%%%%%%%%%%%%%%%%%%%%%%%%%%%%%%%
\begin{frame}[allowframebreaks]{\texttt{PRECOND\_KERKER\_KTF\_MAG}} \label{PRECOND_KERKER_KTF_MAG}
\vspace*{-12pt}
\begin{columns}
\column{0.4\linewidth}
\begin{block}{Type}
Double
\end{block}

\begin{block}{Default}
1.0
\end{block}

\column{0.5\linewidth}
\begin{block}{Unit}
$\textrm{Bohr}^{-1}$
\end{block}

\begin{block}{Example}
\texttt{PRECOND\_KERKER\_KTF\_MAG}: 0.8
\end{block}
\end{columns}

\begin{block}{Description}
The Thomas-Fermi screening length appearing in the \texttt{kerker} preconditioner for the magnetization density (\hyperlink{MIXING_PRECOND_MAG}{\texttt{MIXING\_PRECOND\_MAG}}). 
\end{block}

\end{frame}
%%%%%%%%%%%%%%%%%%%%%%%%%%%%%%%%%%%%%%%%%%%%%%%%%%%%%%%%%%%%%%%%%%%%%%%%%%%%%%%%%%%%%%%%%%%%%



%%%%%%%%%%%%%%%%%%%%%%%%%%%%%%%%%%%%%%%%%%%%%%%%%%%%%%%%%%%%%%%%%%%%%%%%%%%%%%%%%%%%%%%%%%%%%
\begin{frame}[allowframebreaks]{\texttt{PRECOND\_KERKER\_THRESH\_MAG}} \label{PRECOND_KERKER_THRESH_MAG}
\vspace*{-12pt}
\begin{columns}
\column{0.4\linewidth}
\begin{block}{Type}
Double
\end{block}

\begin{block}{Default}
0.1
\end{block}

\column{0.5\linewidth}
\begin{block}{Unit}
No unit
\end{block}

\begin{block}{Example}
\texttt{PRECOND\_KERKER\_THRESH\_MAG}: 0.0
\end{block}
\end{columns}

\begin{block}{Description}
The threshold for the \texttt{kerker} preconditioner the magnetization density (\hyperlink{MIXING_PRECOND_MAG}{\texttt{MIXING\_PRECOND\_MAG}}). 
\end{block}

\begin{block}{Remark}
This threshold will be scaled by the \hyperlink{MIXING_PARAMETER_MAG}{\texttt{MIXING\_PARAMETER\_MAG}}. If the threshold is set to 0, the original \texttt{kerker} preconditioner is recovered.
\end{block}

\end{frame}
%%%%%%%%%%%%%%%%%%%%%%%%%%%%%%%%%%%%%%%%%%%%%%%%%%%%%%%%%%%%%%%%%%%%%%%%%%%%%%%%%%%%%%%%%%%%%



% %%%%%%%%%%%%%%%%%%%%%%%%%%%%%%%%%%%%%%%%%%%%%%%%%%%%%%%%%%%%%%%%%%%%%%%%%%%%%%%%%%%%%%%%%%%%%
% \begin{frame}[allowframebreaks]{\texttt{PRECOND\_RESTA\_Q0}} \label{PRECOND_RESTA_Q0}
% \vspace*{-12pt}
% \begin{columns}
% \column{0.4\linewidth}
% \begin{block}{Type}
% Double
% \end{block}

% \begin{block}{Default}
% 1.36
% \end{block}

% \column{0.5\linewidth}
% \begin{block}{Unit}
% $\textrm{Bohr}^{-1}$
% \end{block}

% \begin{block}{Example}
% \texttt{PRECOND\_RESTA\_Q0}: 1.10
% \end{block}
% \end{columns}

% \begin{block}{Description}
% The Fermi-momentum-related quantity appearing in \texttt{resta} preconditioner (\hyperlink{MIXING_PRECOND}{\texttt{MIXING\_PRECOND}}).
% \end{block}

% \end{frame}
% %%%%%%%%%%%%%%%%%%%%%%%%%%%%%%%%%%%%%%%%%%%%%%%%%%%%%%%%%%%%%%%%%%%%%%%%%%%%%%%%%%%%%%%%%%%%%



% %%%%%%%%%%%%%%%%%%%%%%%%%%%%%%%%%%%%%%%%%%%%%%%%%%%%%%%%%%%%%%%%%%%%%%%%%%%%%%%%%%%%%%%%%%%%%
% \begin{frame}[allowframebreaks]{\texttt{PRECOND\_RESTA\_RS}} \label{PRECOND_RESTA_RS}
% \vspace*{-12pt}
% \begin{columns}
% \column{0.4\linewidth}
% \begin{block}{Type}
% Double
% \end{block}

% \begin{block}{Default}
% 2.76
% \end{block}

% \column{0.5\linewidth}
% \begin{block}{Unit}
% Bohr
% \end{block}

% \begin{block}{Example}
% \texttt{PRECOND\_RESTA\_RS}: 4.28
% \end{block}
% \end{columns}

% \begin{block}{Description}
% The screening length appearing in the \texttt{resta} preconditioner (\hyperlink{MIXING_PRECOND}{\texttt{MIXING\_PRECOND}}). 
% \end{block}

% \end{frame}
% %%%%%%%%%%%%%%%%%%%%%%%%%%%%%%%%%%%%%%%%%%%%%%%%%%%%%%%%%%%%%%%%%%%%%%%%%%%%%%%%%%%%%%%%%%%%%



% %%%%%%%%%%%%%%%%%%%%%%%%%%%%%%%%%%%%%%%%%%%%%%%%%%%%%%%%%%%%%%%%%%%%%%%%%%%%%%%%%%%%%%%%%%%%%
% \begin{frame}[allowframebreaks]{\texttt{PRECOND\_FITPOW}} \label{PRECOND_FITPOW}
% \vspace*{-12pt}
% \begin{columns}
% \column{0.4\linewidth}
% \begin{block}{Type}
% Integer
% \end{block}

% \begin{block}{Default}
% 2
% \end{block}

% \column{0.4\linewidth}
% \begin{block}{Unit}
% No unit
% \end{block}

% \begin{block}{Example}
% \texttt{PRECOND\_FITPOW}: 3
% \end{block}
% \end{columns}

% \begin{block}{Description}
% Half of the highest degree of rational polynomials used for the real-space preconditioning of the SCF iteration. 
% \end{block}

% \begin{block}{Remark}
% Currently this number cannot be larger than 5. Used only for the \texttt{resta} and \texttt{truncated\_kerker} preconditioners.
% \end{block}

% \end{frame}
% %%%%%%%%%%%%%%%%%%%%%%%%%%%%%%%%%%%%%%%%%%%%%%%%%%%%%%%%%%%%%%%%%%%%%%%%%%%%%%%%%%%%%%%%%%%%%

%%%%%%%%%%%%%%%%%%%%%%%%%%%%%%%%%%%%%%%%%%%%%%%%%%%%%%%%%%%%%%%%%%%%%%%%%%%%%%%%%%%%%%%%%%%%%
\begin{frame}[allowframebreaks]{\texttt{TOL\_FOCK}} \label{TOL_FOCK}
\vspace*{-12pt}
\begin{columns}
\column{0.4\linewidth}
\begin{block}{Type}
Double
\end{block}

\begin{block}{Default}
$0.2*$\hyperlink{TOL_SCF}{\texttt{TOL\_SCF}}
\end{block}

\column{0.4\linewidth}
\begin{block}{Unit}
No unit
\end{block}

\begin{block}{Example}
\texttt{TOL\_FOCK}: 1e-6
\end{block}
\end{columns}

\begin{block}{Description}
The tolerance on the Hartree-Fock outer loop, measured by the exact exchange energy difference per atom in 2 consecutive outer loops.
\end{block}

\begin{block}{Remark}
Only active when using hybrid functionals, like PBE0 and HSE. 
\end{block}

\end{frame}
%%%%%%%%%%%%%%%%%%%%%%%%%%%%%%%%%%%%%%%%%%%%%%%%%%%%%%%%%%%%%%%%%%%%%%%%%%%%%%%%%%%%%%%%%%%%%



%%%%%%%%%%%%%%%%%%%%%%%%%%%%%%%%%%%%%%%%%%%%%%%%%%%%%%%%%%%%%%%%%%%%%%%%%%%%%%%%%%%%%%%%%%%%%
\begin{frame}[allowframebreaks]{\texttt{MAXIT\_FOCK}} \label{MAXIT_FOCK}
\vspace*{-12pt}
\begin{columns}
\column{0.4\linewidth}
\begin{block}{Type}
Integer
\end{block}

\begin{block}{Default}
20
\end{block}

\column{0.4\linewidth}
\begin{block}{Unit}
No unit
\end{block}

\begin{block}{Example}
\texttt{MAXIT\_FOCK}: 50
\end{block}
\end{columns}

\begin{block}{Description}
The maximum number of iterations for Hartree-Fock outer loop.
\end{block}

\begin{block}{Remark}
Only active when using hybrid functionals, like PBE0 and HSE. 
\end{block}

\end{frame}
%%%%%%%%%%%%%%%%%%%%%%%%%%%%%%%%%%%%%%%%%%%%%%%%%%%%%%%%%%%%%%%%%%%%%%%%%%%%%%%%%%%%%%%%%%%%%

%%%%%%%%%%%%%%%%%%%%%%%%%%%%%%%%%%%%%%%%%%%%%%%%%%%%%%%%%%%%%%%%%%%%%%%%%%%%%%%%%%%%%%%%%%%%%
\begin{frame}[allowframebreaks]{\texttt{MINIT\_FOCK}} \label{MINIT_FOCK}
\vspace*{-12pt}
\begin{columns}
\column{0.4\linewidth}
\begin{block}{Type}
Integer
\end{block}

\begin{block}{Default}
2
\end{block}

\column{0.4\linewidth}
\begin{block}{Unit}
No unit
\end{block}

\begin{block}{Example}
\texttt{MINIT\_FOCK}: 3
\end{block}
\end{columns}

\begin{block}{Description}
The minimum number of iterations for Hartree-Fock outer loop.
\end{block}

\begin{block}{Remark}
Only active when using hybrid functionals, like PBE0 and HSE. 
\end{block}

\end{frame}
%%%%%%%%%%%%%%%%%%%%%%%%%%%%%%%%%%%%%%%%%%%%%%%%%%%%%%%%%%%%%%%%%%%%%%%%%%%%%%%%%%%%%%%%%%%%%


%%%%%%%%%%%%%%%%%%%%%%%%%%%%%%%%%%%%%%%%%%%%%%%%%%%%%%%%%%%%%%%%%%%%%%%%%%%%%%%%%%%%%%%%%%%%%
\begin{frame}[allowframebreaks]{\texttt{TOL\_SCF\_INIT}} \label{TOL_SCF_INIT}
\vspace*{-12pt}
\begin{columns}
\column{0.4\linewidth}
\begin{block}{Type}
Double
\end{block}

\begin{block}{Default}
$\max($\hyperlink{TOL_FOCK}{\texttt{TOL\_FOCK}}$\times 10,0.001)$
\end{block}

\column{0.4\linewidth}
\begin{block}{Unit}
No unit
\end{block}

\begin{block}{Example}
\texttt{TOL\_SCF\_INIT}: 1e-6
\end{block}
\end{columns}

\begin{block}{Description}
The initial SCF tolerance for PBE iteration when using hybrid functionals. 
\end{block}

\begin{block}{Remark}
Only active when using hybrid functionals, like PBE0 and HSE. Change the \texttt{TOL\_SCF\_INIT} to change the initial guess for Hartree Fock outer loop.
\end{block}

\end{frame}
%%%%%%%%%%%%%%%%%%%%%%%%%%%%%%%%%%%%%%%%%%%%%%%%%%%%%%%%%%%%%%%%%%%%%%%%%%%%%%%%%%%%%%%%%%%%%

%%%%%%%%%%%%%%%%%%%%%%%%%%%%%%%%%%%%%%%%%%%%%%%%%%%%%%%%%%%%%%%%%%%%%%%%%%%%%%%%%%%%%%%%%%%%%
\begin{frame}[allowframebreaks]{\texttt{ACE\_FLAG}} \label{ACE_FLAG}
\vspace*{-12pt}
\begin{columns}
\column{0.4\linewidth}
\begin{block}{Type}
Integer
\end{block}

\begin{block}{Default}
1
\end{block}

\column{0.4\linewidth}
\begin{block}{Unit}
No unit
\end{block}

\begin{block}{Example}
\texttt{ACE\_FLAG}: 0
\end{block}
\end{columns}

\begin{block}{Description}
Use ACE operator to accelarte the hybrid calculation.
\end{block}

\begin{block}{Remark}
Without ACE operator, the hybrid calculation will be way slower than with it on depending on the system size.
\end{block}

\end{frame}
%%%%%%%%%%%%%%%%%%%%%%%%%%%%%%%%%%%%%%%%%%%%%%%%%%%%%%%%%%%%%%%%%%%%%%%%%%%%%%%%%%%%%%%%%%%%%


%%%%%%%%%%%%%%%%%%%%%%%%%%%%%%%%%%%%%%%%%%%%%%%%%%%%%%%%%%%%%%%%%%%%%%%%%%%%%%%%%%%%%%%%%%%%%
\begin{frame}[allowframebreaks]{\texttt{EXX\_METHOD}} \label{EXX_METHOD}
\vspace*{-12pt}
\begin{columns}
\column{0.4\linewidth}
\begin{block}{Type}
String
\end{block}

\begin{block}{Default}
\texttt{FOURIER\_SPACE}
\end{block}

\column{0.4\linewidth}
\begin{block}{Unit}
No unit
\end{block}

\begin{block}{Example}
\texttt{EXX\_METHOD}: \texttt{REAL\_SPACE}
\end{block}
\end{columns}

\begin{block}{Description}
Methods to solve Poisson's equation in Exact Exchange part. Options include using FFT to solve it in Fourier space and using linear solver, like CG, to solve in Real space.
\end{block}

\begin{block}{Remark}
Only active when using hybrid functionals for molecule simulation, like PBE0 and HSE. 
\texttt{FOURIER\_SPACE} method is much faster than \texttt{REAL\_SPACE} method.
\end{block}

\end{frame}
%%%%%%%%%%%%%%%%%%%%%%%%%%%%%%%%%%%%%%%%%%%%%%%%%%%%%%%%%%%%%%%%%%%%%%%%%%%%%%%%%%%%%%%%%%%%%

%%%%%%%%%%%%%%%%%%%%%%%%%%%%%%%%%%%%%%%%%%%%%%%%%%%%%%%%%%%%%%%%%%%%%%%%%%%%%%%%%%%%%%%%%%%%%
\begin{frame}[allowframebreaks]{\texttt{EXX\_MEM}} \label{EXX_MEM}
\vspace*{-12pt}
\begin{columns}
\column{0.4\linewidth}
\begin{block}{Type}
Integer
\end{block}

\begin{block}{Default}
0
\end{block}

\column{0.4\linewidth}
\begin{block}{Unit}
No unit
\end{block}

\begin{block}{Example}
\texttt{EXX\_MEM}: 1
\end{block}
\end{columns}

\begin{block}{Description}
Number of Poisson's equations to be solved in each process at a time when creating exact exchange operator or ACE operator. It could be any non negative integer. 
\end{block}

\begin{block}{Remark}
If set to 0, user could get fastest speed but with highest memory requirement. If set to positive integer, user could run large systems with relatively small memory but slower speed. 
\end{block}

\end{frame}
%%%%%%%%%%%%%%%%%%%%%%%%%%%%%%%%%%%%%%%%%%%%%%%%%%%%%%%%%%%%%%%%%%%%%%%%%%%%%%%%%%%%%%%%%%%%%


%%%%%%%%%%%%%%%%%%%%%%%%%%%%%%%%%%%%%%%%%%%%%%%%%%%%%%%%%%%%%%%%%%%%%%%%%%%%%%%%%%%%%%%%%%%%%
\begin{frame}[allowframebreaks]{\texttt{EXX\_ACE\_VALENCE\_STATES}} \label{EXX_ACE_VALENCE_STATES}
\vspace*{-12pt}
\begin{columns}
\column{0.4\linewidth}
\begin{block}{Type}
Integer
\end{block}

\begin{block}{Default}
3
\end{block}

\column{0.4\linewidth}
\begin{block}{Unit}
No unit
\end{block}

\begin{block}{Example}
\texttt{EXX\_ACE\_VALENCE\_STATES}: 1
\end{block}
\end{columns}

\begin{block}{Description}
Control of number of unoccupied states used to construct ACE operator.
\end{block}

\begin{block}{Remark}
Only active when using hybrid functionals with ACE operator. 
\end{block}

\end{frame}
%%%%%%%%%%%%%%%%%%%%%%%%%%%%%%%%%%%%%%%%%%%%%%%%%%%%%%%%%%%%%%%%%%%%%%%%%%%%%%%%%%%%%%%%%%%%%


%%%%%%%%%%%%%%%%%%%%%%%%%%%%%%%%%%%%%%%%%%%%%%%%%%%%%%%%%%%%%%%%%%%%%%%%%%%%%%%%%%%%%%%%%%%%%
\begin{frame}[allowframebreaks]{\texttt{EXX\_DOWNSAMPLING}} \label{EXX_DOWNSAMPLING}
\vspace*{-12pt}
\begin{columns}
\column{0.4\linewidth}
\begin{block}{Type}
Integer
\end{block}

\begin{block}{Default}
1 1 1
\end{block}

\column{0.4\linewidth}
\begin{block}{Unit}
No unit
\end{block}

\begin{block}{Example}
\texttt{EXX\_DOWNSAMPLING}: 1 2 3
\end{block}
\end{columns}

\begin{block}{Description}
Down-sampling of k-points grids. There should be 3 nonnegative integers. 0 means using 0 k-point in that direction, 
requiring 0 is one of the k-point after time-reversal symmetry in that direction. 
Positive value should be a factor of the number of grid points in that direction. 
\end{block}

\end{frame}
%%%%%%%%%%%%%%%%%%%%%%%%%%%%%%%%%%%%%%%%%%%%%%%%%%%%%%%%%%%%%%%%%%%%%%%%%%%%%%%%%%%%%%%%%%%%%


%%%%%%%%%%%%%%%%%%%%%%%%%%%%%%%%%%%%%%%%%%%%%%%%%%%%%%%%%%%%%%%%%%%%%%%%%%%%%%%%%%%%%%%%%%%%%
\begin{frame}[allowframebreaks]{\texttt{EXX\_DIVERGENCE}} \label{EXX_DIVERGENCE}
\vspace*{-12pt}
\begin{columns}
\column{0.35\linewidth}
\begin{block}{Type}
String
\end{block}

\begin{block}{Default}
SPHERICAL
\end{block}

\column{0.55\linewidth}
\begin{block}{Unit}
No unit
\end{block}

\begin{block}{Example}
\texttt{EXX\_DIVERGENCE}: \texttt{AUXILIARY}
\end{block}
\end{columns}

\begin{block}{Description}
Treatment of divergence in exact exchange. Options are \texttt{SPHERICAL} (spherical truncation), 
\texttt{AUXILIARY} (auxiliary function method) and \texttt{ERFC} (erfc screening).
\end{block}

\begin{block}{Remark}
For systems with cube-like geometry, both methods converge fast. For slab and wire, auxiliary function method is a better option. 
ERFC screening is the default option for HSE in bulk and molecule simulation.
\end{block}

\end{frame}
%%%%%%%%%%%%%%%%%%%%%%%%%%%%%%%%%%%%%%%%%%%%%%%%%%%%%%%%%%%%%%%%%%%%%%%%%%%%%%%%%%%%%%%%%%%%%


%%%%%%%%%%%%%%%%%%%%%%%%%%%%%%%%%%%%%%%%%%%%%%%%%%%%%%%%%%%%%%%%%%%%%%%%%%%%%%%%%%%%%%%%%%%%%
\begin{frame}[allowframebreaks,c]{} \label{Electrostatics}

\begin{center}
\Huge \textbf{Electrostatics}
\end{center}

\end{frame}
%%%%%%%%%%%%%%%%%%%%%%%%%%%%%%%%%%%%%%%%%%%%%%%%%%%%%%%%%%%%%%%%%%%%%%%%%%%%%%%%%%%%%%%%%%%%%




%%%%%%%%%%%%%%%%%%%%%%%%%%%%%%%%%%%%%%%%%%%%%%%%%%%%%%%%%%%%%%%%%%%%%%%%%%%%%%%%%%%%%%%%%%%%%
\begin{frame}[allowframebreaks]{\texttt{TOL\_POISSON}} \label{TOL_POISSON}
\vspace*{-12pt}
\begin{columns}
\column{0.4\linewidth}
\begin{block}{Type}
Integer
\end{block}

\begin{block}{Default}
\hyperlink{TOL_SCF}{\texttt{TOL\_SCF}}$\times 0.01$
\end{block}

\column{0.4\linewidth}
\begin{block}{Unit}
No unit
\end{block}

\begin{block}{Example}
\texttt{TOL\_POISSON}: 1e-6
\end{block}
\end{columns}

\begin{block}{Description}
The tolerance on the norm of the relative residual for the Poisson equation.
\end{block}

\begin{block}{Remark}
The tolerance for poisson solver should not be worse than \hyperlink{TOL_SCF}{\texttt{TOL\_SCF}}, otherwise it might seriously affect  the convergence of the SCF iteration.
\end{block}

\end{frame}
%%%%%%%%%%%%%%%%%%%%%%%%%%%%%%%%%%%%%%%%%%%%%%%%%%%%%%%%%%%%%%%%%%%%%%%%%%%%%%%%%%%%%%%%%%%%%



%%%%%%%%%%%%%%%%%%%%%%%%%%%%%%%%%%%%%%%%%%%%%%%%%%%%%%%%%%%%%%%%%%%%%%%%%%%%%%%%%%%%%%%%%%%%%
\begin{frame}[allowframebreaks]{\texttt{TOL\_PSEUDOCHARGE}} \label{TOL_PSEUDOCHARGE}
\vspace*{-12pt}
\begin{columns}
\column{0.4\linewidth}
\begin{block}{Type}
Double
\end{block}

\begin{block}{Default}
\hyperlink{TOL_SCF}{\texttt{TOL\_SCF}}$\times 0.001$
\end{block}

\column{0.45\linewidth}
\begin{block}{Unit}
No unit
\end{block}

\begin{block}{Example}
\texttt{TOL\_PSEUDOCHARGE}: 1e-6
\end{block}
\end{columns}

\begin{block}{Description}
The normalized error in the net enclosed charge for the pseudocharge density of each atom.
\end{block}

\end{frame}
%%%%%%%%%%%%%%%%%%%%%%%%%%%%%%%%%%%%%%%%%%%%%%%%%%%%%%%%%%%%%%%%%%%%%%%%%%%%%%%%%%%%%%%%%%%%%




%%%%%%%%%%%%%%%%%%%%%%%%%%%%%%%%%%%%%%%%%%%%%%%%%%%%%%%%%%%%%%%%%%%%%%%%%%%%%%%%%%%%%%%%%%%%%
\begin{frame}[allowframebreaks]{\texttt{REFERENCE\_CUTOFF}} \label{REFERENCE_CUTOFF}
\vspace*{-12pt}
\begin{columns}
\column{0.4\linewidth}
\begin{block}{Type}
Double
\end{block}

\begin{block}{Default}
0.5
\end{block}

\column{0.4\linewidth}
\begin{block}{Unit}
Bohr
\end{block}

\begin{block}{Example}
\texttt{REFERENCE\_CUTOFF}: 1.0
\end{block}
\end{columns}

\begin{block}{Description}
The cutoff radius of the reference potential used for evaluating the electrostatic correction arising from overlapping pseudocharge densities.
\end{block}

\begin{block}{Remark}
This number should be smaller than half the smallest interatomic distance.
\end{block}

\end{frame}
%%%%%%%%%%%%%%%%%%%%%%%%%%%%%%%%%%%%%%%%%%%%%%%%%%%%%%%%%%%%%%%%%%%%%%%%%%%%%%%%%%%%%%%%%%%%%


%%%%%%%%%%%%%%%%%%%%%%%%%%%%%%%%%%%%%%%%%%%%%%%%%%%%%%%%%%%%%%%%%%%%%%%%%%%%%%%%%%%%%%%%%%%%%
\begin{frame}[allowframebreaks,c]{} \label{Stress calculation}

\begin{center}
\Huge \textbf{Stress calculation}
\end{center}

\end{frame}
%%%%%%%%%%%%%%%%%%%%%%%%%%%%%%%%%%%%%%%%%%%%%%%%%%%%%%%%%%%%%%%%%%%%%%%%%%%%%%%%%%%%%%%%%%%%%





%%%%%%%%%%%%%%%%%%%%%%%%%%%%%%%%%%%%%%%%%%%%%%%%%%%%%%%%%%%%%%%%%%%%%%%%%%%%%%%%%%%%%%%%%%%%%
\begin{frame}[allowframebreaks]{\texttt{CALC\_STRESS}} \label{CALC_STRESS}
\vspace*{-12pt}
\begin{columns}
\column{0.4\linewidth}
\begin{block}{Type}
0 or 1
\end{block}

\begin{block}{Default}
0
\end{block}

\column{0.4\linewidth}
\begin{block}{Unit}
No unit
\end{block}

\begin{block}{Example}
\texttt{CALC\_STRESS}: 1
\end{block}
\end{columns}

\begin{block}{Description}
Flag for calculation of the Hellmann-Feynman stress tensor (in cartesian coordinates).
\end{block}

%\begin{block}{Remark}
%\end{block}

\end{frame}
%%%%%%%%%%%%%%%%%%%%%%%%%%%%%%%%%%%%%%%%%%%%%%%%%%%%%%%%%%%%%%%%%%%%%%%%%%%%%%%%%%%%%%%%%%%%%


%%%%%%%%%%%%%%%%%%%%%%%%%%%%%%%%%%%%%%%%%%%%%%%%%%%%%%%%%%%%%%%%%%%%%%%%%%%%%%%%%%%%%%%%%%%%%
\begin{frame}[allowframebreaks]{\texttt{CALC\_PRES}} \label{CALC_PRES}
\vspace*{-12pt}
\begin{columns}
\column{0.4\linewidth}
\begin{block}{Type}
0 or 1
\end{block}

\begin{block}{Default}
0
\end{block}

\column{0.4\linewidth}
\begin{block}{Unit}
No unit
\end{block}

\begin{block}{Example}
\texttt{CALC\_PRES}: 1
\end{block}
\end{columns}

\begin{block}{Description}
Flag for calculation of the pressure.
\end{block}

\begin{block}{Remark}
Pressure is directly calculated, without calculation of the stress tensor.
\end{block}

\end{frame}
%%%%%%%%%%%%%%%%%%%%%%%%%%%%%%%%%%%%%%%%%%%%%%%%%%%%%%%%%%%%%%%%%%%%%%%%%%%%%%%%%%%%%%%%%%%%%



%%%%%%%%%%%%%%%%%%%%%%%%%%%%%%%%%%%%%%%%%%%%%%%%%%%%%%%%%%%%%%%%%%%%%%%%%%%%%%%%%%%%%%%%%%%%%
\begin{frame}[allowframebreaks,c]{} \label{MD}

\begin{center}
\Huge \textbf{MD}
\end{center}

\end{frame}
%%%%%%%%%%%%%%%%%%%%%%%%%%%%%%%%%%%%%%%%%%%%%%%%%%%%%%%%%%%%%%%%%%%%%%%%%%%%%%%%%%%%%%%%%%%%%





%%%%%%%%%%%%%%%%%%%%%%%%%%%%%%%%%%%%%%%%%%%%%%%%%%%%%%%%%%%%%%%%%%%%%%%%%%%%%%%%%%%%%%%%%%%%%
\begin{frame}[allowframebreaks]{\texttt{MD\_FLAG}} \label{MD_FLAG}
\vspace*{-12pt}
\begin{columns}
\column{0.4\linewidth}
\begin{block}{Type}
0 or 1
\end{block}

\begin{block}{Default}
0
\end{block}

\column{0.4\linewidth}
\begin{block}{Unit}
No unit
\end{block}

\begin{block}{Example}
\texttt{MD\_FLAG}: 1
\end{block}
\end{columns}

\begin{block}{Description}
MD simulations are performed if the flag is set to 1.
\end{block}

\begin{block}{Remark}
\hyperlink{MD_FLAG}{\texttt{MD\_FLAG}} and \hyperlink{RELAX_FLAG}{\texttt{RELAX\_FLAG}} both cannot be set to 1. 
\end{block}

\end{frame}
%%%%%%%%%%%%%%%%%%%%%%%%%%%%%%%%%%%%%%%%%%%%%%%%%%%%%%%%%%%%%%%%%%%%%%%%%%%%%%%%%%%%%%%%%%%%%



%%%%%%%%%%%%%%%%%%%%%%%%%%%%%%%%%%%%%%%%%%%%%%%%%%%%%%%%%%%%%%%%%%%%%%%%%%%%%%%%%%%%%%%%%%%%%
\begin{frame}[allowframebreaks]{\texttt{MD\_METHOD}} \label{MD_METHOD}
\vspace*{-12pt}
\begin{columns}
\column{0.4\linewidth}
\begin{block}{Type}
String
\end{block}

\begin{block}{Default}
NVE
\end{block}

\column{0.4\linewidth}
\begin{block}{Unit}
No unit
\end{block}

\begin{block}{Example}
\texttt{MD\_METHOD}: NVE
\end{block}
\end{columns}

\begin{block}{Description}
Type of MD to be performed. 
\end{block}

\begin{block}{Remark}
Only NVE (microcanonical ensemble) is supported.
\end{block}

\end{frame}
%%%%%%%%%%%%%%%%%%%%%%%%%%%%%%%%%%%%%%%%%%%%%%%%%%%%%%%%%%%%%%%%%%%%%%%%%%%%%%%%%%%%%%%%%%%%%



%%%%%%%%%%%%%%%%%%%%%%%%%%%%%%%%%%%%%%%%%%%%%%%%%%%%%%%%%%%%%%%%%%%%%%%%%%%%%%%%%%%%%%%%%%%%%
\begin{frame}[allowframebreaks]{\texttt{MD\_NSTEP}} \label{MD_NSTEP}
\vspace*{-12pt}
\begin{columns}
\column{0.4\linewidth}
\begin{block}{Type}
Integer
\end{block}

\begin{block}{Default}
0
\end{block}

\column{0.4\linewidth}
\begin{block}{Unit}
No unit
\end{block}

\begin{block}{Example}
\texttt{MD\_NSTEP}: 100
\end{block}
\end{columns}

\begin{block}{Description}
Specifies the number of MD steps.
\end{block}

\begin{block}{Remark}
If MD\_NSTEP $= N$, the MD runs from $0$ to $(N-1) \times$ \hyperlink{MD_TIMESTEP}{\texttt{MD\_TIMESTEP}} fs.  
\end{block}

\end{frame}
%%%%%%%%%%%%%%%%%%%%%%%%%%%%%%%%%%%%%%%%%%%%%%%%%%%%%%%%%%%%%%%%%%%%%%%%%%%%%%%%%%%%%%%%%%%%%


%%%%%%%%%%%%%%%%%%%%%%%%%%%%%%%%%%%%%%%%%%%%%%%%%%%%%%%%%%%%%%%%%%%%%%%%%%%%%%%%%%%%%%%%%%%%%
\begin{frame}[allowframebreaks]{\texttt{MD\_TIMESTEP}} \label{MD_TIMESTEP}
\vspace*{-12pt}
\begin{columns}
\column{0.4\linewidth}
\begin{block}{Type}
Double
\end{block}

\begin{block}{Default}
1
\end{block}

\column{0.4\linewidth}
\begin{block}{Unit}
Femtosecond
\end{block}

\begin{block}{Example}
\texttt{MD\_TIMESTEP}: 0.1
\end{block}
\end{columns}

\begin{block}{Description}
MD time step. 
\end{block}

\begin{block}{Remark}
Total MD time is given by:  \hyperlink{MD_TIMESTEP}{\texttt{MD\_TIMESTEP}} $\times$ \hyperlink{MD_NSTEP}{\texttt{MD\_NSTEP}}.
\end{block}

\end{frame}
%%%%%%%%%%%%%%%%%%%%%%%%%%%%%%%%%%%%%%%%%%%%%%%%%%%%%%%%%%%%%%%%%%%%%%%%%%%%%%%%%%%%%%%%%%%%%


%%%%%%%%%%%%%%%%%%%%%%%%%%%%%%%%%%%%%%%%%%%%%%%%%%%%%%%%%%%%%%%%%%%%%%%%%%%%%%%%%%%%%%%%%%%%%
\begin{frame}[allowframebreaks]{\texttt{ION\_TEMP}} \label{ION_TEMP}
\vspace*{-12pt}
\begin{columns}
\column{0.4\linewidth}
\begin{block}{Type}
Double
\end{block}

\begin{block}{Default}
No Default
\end{block}

\column{0.4\linewidth}
\begin{block}{Unit}
Kelvin
\end{block}

\begin{block}{Example}
\texttt{ION\_TEMP}: 315
\end{block}
\end{columns}

\begin{block}{Description}
Starting ionic temperature in MD, used to generate initial velocity distribution.
\end{block}

\begin{block}{Remark}
Must be specified if \hyperlink{MD_FLAG}{\texttt{MD\_FLAG}} is set to $1$.
\end{block}

\end{frame}
%%%%%%%%%%%%%%%%%%%%%%%%%%%%%%%%%%%%%%%%%%%%%%%%%%%%%%%%%%%%%%%%%%%%%%%%%%%%%%%%%%%%%%%%%%%%%



%%%%%%%%%%%%%%%%%%%%%%%%%%%%%%%%%%%%%%%%%%%%%%%%%%%%%%%%%%%%%%%%%%%%%%%%%%%%%%%%%%%%%%%%%%%%%
\begin{frame}[allowframebreaks]{\texttt{ION\_ELEC\_EQT}} \label{ION_ELEC_EQT}
\vspace*{-12pt}
\begin{columns}
\column{0.4\linewidth}
\begin{block}{Type}
Integer
\end{block}

\begin{block}{Default}
1
\end{block}

\column{0.4\linewidth}
\begin{block}{Unit}
No unit
\end{block}

\begin{block}{Example}
\texttt{ION\_ELEC\_EQT}: 0
\end{block}
\end{columns}

\begin{block}{Description}
Flag that determines whether the \hyperlink{ELEC_TEMP}{\texttt{ELEC\_TEMP}} will be set equal to \hyperlink{ION_TEMP}{\texttt{ION\_TEMP}} during MD.
\end{block}

\begin{block}{Remark}
If the flag is set to 0, the values of \hyperlink{ELEC_TEMP}{\texttt{ELEC\_TEMP}} and \hyperlink{ION_TEMP}{\texttt{ION\_TEMP}} need to be identical.  
\end{block}

\end{frame}
%%%%%%%%%%%%%%%%%%%%%%%%%%%%%%%%%%%%%%%%%%%%%%%%%%%%%%%%%%%%%%%%%%%%%%%%%%%%%%%%%%%%%%%%%%%%%



%%%%%%%%%%%%%%%%%%%%%%%%%%%%%%%%%%%%%%%%%%%%%%%%%%%%%%%%%%%%%%%%%%%%%%%%%%%%%%%%%%%%%%%%%%%%%
\begin{frame}[allowframebreaks]{\texttt{RESTART\_FLAG}} \label{RESTART_FLAG}
\vspace*{-12pt}
\begin{columns}
\column{0.4\linewidth}
\begin{block}{Type}
0 or 1
\end{block}

\begin{block}{Default}
0
\end{block}

\column{0.4\linewidth}
\begin{block}{Unit}
No unit
\end{block}

\begin{block}{Example}
\texttt{RESTART\_FLAG}: 0
\end{block}
\end{columns}

\begin{block}{Description}
Flag for restarting molecular dynamics and structural relaxation.
\end{block}

\begin{block}{Remark}
Restarts from the previous configuration which is stored in a .restart file.
\end{block}

\end{frame}
%%%%%%%%%%%%%%%%%%%%%%%%%%%%%%%%%%%%%%%%%%%%%%%%%%%%%%%%%%%%%%%%%%%%%%%%%%%%%%%%%%%%%%%%%%%%%


%%%%%%%%%%%%%%%%%%%%%%%%%%%%%%%%%%%%%%%%%%%%%%%%%%%%%%%%%%%%%%%%%%%%%%%%%%%%%%%%%%%%%%%%%%%%%
\begin{frame}[allowframebreaks,c]{} \label{Structural relaxation}

\begin{center}
\Huge \textbf{Structural relaxation}
\end{center}

\end{frame}
%%%%%%%%%%%%%%%%%%%%%%%%%%%%%%%%%%%%%%%%%%%%%%%%%%%%%%%%%%%%%%%%%%%%%%%%%%%%%%%%%%%%%%%%%%%%%





%%%%%%%%%%%%%%%%%%%%%%%%%%%%%%%%%%%%%%%%%%%%%%%%%%%%%%%%%%%%%%%%%%%%%%%%%%%%%%%%%%%%%%%%%%%%%
\begin{frame}[allowframebreaks]{\texttt{RELAX\_FLAG}} \label{RELAX_FLAG}
\vspace*{-12pt}
\begin{columns}
\column{0.4\linewidth}
\begin{block}{Type}
0 or 1
\end{block}

\begin{block}{Default}
0
\end{block}

\column{0.4\linewidth}
\begin{block}{Unit}
No unit
\end{block}

\begin{block}{Example}
\texttt{RELAX\_FLAG}: 1
\end{block}
\end{columns}

\begin{block}{Description}
Flag for performing structural relaxation. $0$ means no structural relaxation. $1$ represents relaxation of atom positions. $2$ represents optimization of volume with the fractional coordinates of the atoms fixed.
\end{block}

\begin{block}{Remark}
This flag should not be specified if \hyperlink{MD_FLAG}{\texttt{MD\_FLAG}} is set to $1$. 
\end{block}

\end{frame}
%%%%%%%%%%%%%%%%%%%%%%%%%%%%%%%%%%%%%%%%%%%%%%%%%%%%%%%%%%%%%%%%%%%%%%%%%%%%%%%%%%%%%%%%%%%%%


%%%%%%%%%%%%%%%%%%%%%%%%%%%%%%%%%%%%%%%%%%%%%%%%%%%%%%%%%%%%%%%%%%%%%%%%%%%%%%%%%%%%%%%%%%%%%
\begin{frame}[allowframebreaks]{\texttt{RELAX\_METHOD}} \label{RELAX_METHOD}
\vspace*{-12pt}
\begin{columns}
\column{0.4\linewidth}
\begin{block}{Type}
String
\end{block}

\begin{block}{Default}
LBFGS
\end{block}

\column{0.4\linewidth}
\begin{block}{Unit}
No unit
\end{block}

\begin{block}{Example}
\texttt{RELAX\_METHOD}: NLCG
\end{block}
\end{columns}

\begin{block}{Description}
Specifies the algorithm for structural relaxation. The choices are `LBFGS' (limited-memory BFGS), `NLCG' (Non-linear conjugate gradient), and `FIRE' (Fast inertial relaxation engine). 
\end{block}

\begin{block}{Remark}
LBFGS is typically the best choice.
\end{block}

\end{frame}
%%%%%%%%%%%%%%%%%%%%%%%%%%%%%%%%%%%%%%%%%%%%%%%%%%%%%%%%%%%%%%%%%%%%%%%%%%%%%%%%%%%%%%%%%%%%%




%%%%%%%%%%%%%%%%%%%%%%%%%%%%%%%%%%%%%%%%%%%%%%%%%%%%%%%%%%%%%%%%%%%%%%%%%%%%%%%%%%%%%%%%%%%%%
\begin{frame}[allowframebreaks]{\texttt{RELAX\_NITER}} \label{RELAX_NITER}
\vspace*{-12pt}
\begin{columns}
\column{0.4\linewidth}
\begin{block}{Type}
Integer
\end{block}

\begin{block}{Default}
100
\end{block}

\column{0.4\linewidth}
\begin{block}{Unit}
No unit
\end{block}

\begin{block}{Example}
\texttt{RELAX\_NITER}: 25
\end{block}
\end{columns}

\begin{block}{Description}
Specifies the maximum number of iterations for the structural relaxation (\hyperlink{RELAX_FLAG}{\texttt{RELAX\_FLAG}}).
\end{block}

\begin{block}{Remark}
If \hyperlink{RESTART_FLAG}{\texttt{RESTART\_FLAG}} is set to $1$, then relaxation will restart from the last atomic configuration and run for maximum of \hyperlink{RELAX_NITER}{\texttt{RELAX\_NITER}} iterations. 
\end{block}

\end{frame}
%%%%%%%%%%%%%%%%%%%%%%%%%%%%%%%%%%%%%%%%%%%%%%%%%%%%%%%%%%%%%%%%%%%%%%%%%%%%%%%%%%%%%%%%%%%%%

%%%%%%%%%%%%%%%%%%%%%%%%%%%%%%%%%%%%%%%%%%%%%%%%%%%%%%%%%%%%%%%%%%%%%%%%%%%%%%%%%%%%%%%%%%%%%
\begin{frame}[allowframebreaks]{\texttt{TOL\_RELAX}} \label{TOL_RELAX}
\vspace*{-12pt}
\begin{columns}
\column{0.4\linewidth}
\begin{block}{Type}
Double
\end{block}

\begin{block}{Default}
5e-4
\end{block}

\column{0.4\linewidth}
\begin{block}{Unit}
Ha/Bohr
\end{block}

\begin{block}{Example}
\texttt{TOL\_RELAX}: 1e-3
\end{block}
\end{columns}

\begin{block}{Description}
Specifies the tolerance for termination of the structural relaxation. The tolerance is defined on the maximum force component (in absolute sense) over all atoms. 
\end{block}

%\begin{block}{Remark}
%\end{block}

\end{frame}
%%%%%%%%%%%%%%%%%%%%%%%%%%%%%%%%%%%%%%%%%%%%%%%%%%%%%%%%%%%%%%%%%%%%%%%%%%%%%%%%%%%%%%%%%%%%%


%%%%%%%%%%%%%%%%%%%%%%%%%%%%%%%%%%%%%%%%%%%%%%%%%%%%%%%%%%%%%%%%%%%%%%%%%%%%%%%%%%%%%%%%%%%%%
\begin{frame}[allowframebreaks]{\texttt{TOL\_RELAX\_CELL}} \label{TOL_RELAX_CELL}
\vspace*{-12pt}
\begin{columns}
\column{0.4\linewidth}
\begin{block}{Type}
Double
\end{block}

\begin{block}{Default}
1e-2
\end{block}

\column{0.4\linewidth}
\begin{block}{Unit}
GPa
\end{block}

\begin{block}{Example}
\texttt{TOL\_RELAX}: 1e-3
\end{block}
\end{columns}

\begin{block}{Description}
Specifies the tolerance for termination of the cell relaxation. The tolerance is defined on the maximum principle stress component. 
\end{block}

%\begin{block}{Remark}
%\end{block}

\end{frame}
%%%%%%%%%%%%%%%%%%%%%%%%%%%%%%%%%%%%%%%%%%%%%%%%%%%%%%%%%%%%%%%%%%%%%%%%%%%%%%%%%%%%%%%%%%%%%


%%%%%%%%%%%%%%%%%%%%%%%%%%%%%%%%%%%%%%%%%%%%%%%%%%%%%%%%%%%%%%%%%%%%%%%%%%%%%%%%%%%%%%%%%%%%%
\begin{frame}[allowframebreaks]{\texttt{RELAX\_MAXDIAL}} \label{RELAX_MAXDIAL}
\vspace*{-12pt}
\begin{columns}
\column{0.4\linewidth}
\begin{block}{Type}
Double
\end{block}

\begin{block}{Default}
1.2
\end{block}

\column{0.4\linewidth}
\begin{block}{Unit}
No unit
\end{block}

\begin{block}{Example}
\texttt{RELAX\_MAXDIAL}: 1.4
\end{block}
\end{columns}

\begin{block}{Description}
The maximum scaling of the volume allowed with respect to the initial volume defined by \hyperlink{CELL}{\texttt{CELL}} and \hyperlink{LATVEC}{\texttt{LATVEC}}. This will determine the upper-bound and lower-bound in the bisection method (Brent's method) for the volume optimization.
\end{block}

%\begin{block}{Remark}
%\end{block}

\end{frame}
%%%%%%%%%%%%%%%%%%%%%%%%%%%%%%%%%%%%%%%%%%%%%%%%%%%%%%%%%%%%%%%%%%%%%%%%%%%%%%%%%%%%%%%%%%%%%



%%%%%%%%%%%%%%%%%%%%%%%%%%%%%%%%%%%%%%%%%%%%%%%%%%%%%%%%%%%%%%%%%%%%%%%%%%%%%%%%%%%%%%%%%%%%%
\begin{frame}[allowframebreaks]{\texttt{NLCG\_SIGMA}} \label{NLCG_SIGMA}
\vspace*{-12pt}
\begin{columns}
\column{0.4\linewidth}
\begin{block}{Type}
Double
\end{block}

\begin{block}{Default}
0.5
\end{block}

\column{0.4\linewidth}
\begin{block}{Unit}
No unit
\end{block}

\begin{block}{Example}
\texttt{NLCG\_SIGMA}: 1
\end{block}
\end{columns}

\begin{block}{Description}
Parameter in the secant method used to control the step length in NLCG (\hyperlink{RELAX_METHOD}{\texttt{RELAX\_METHOD}}). 
\end{block}

\begin{block}{Remark}
Default value works well in most cases.
\end{block}

\end{frame}
%%%%%%%%%%%%%%%%%%%%%%%%%%%%%%%%%%%%%%%%%%%%%%%%%%%%%%%%%%%%%%%%%%%%%%%%%%%%%%%%%%%%%%%%%%%%%


%%%%%%%%%%%%%%%%%%%%%%%%%%%%%%%%%%%%%%%%%%%%%%%%%%%%%%%%%%%%%%%%%%%%%%%%%%%%%%%%%%%%%%%%%%%%%
\begin{frame}[allowframebreaks]{\texttt{L\_HISTORY}} \label{L_HISTORY}
\vspace*{-12pt}
\begin{columns}
\column{0.4\linewidth}
\begin{block}{Type}
Integer
\end{block}

\begin{block}{Default}
20
\end{block}

\column{0.4\linewidth}
\begin{block}{Unit}
No unit
\end{block}

\begin{block}{Example}
\texttt{L\_HISTORY}: 15
\end{block}
\end{columns}

\begin{block}{Description}
Size of history in LBFGS (\hyperlink{RELAX_METHOD}{\texttt{RELAX\_METHOD}}).
\end{block}

\begin{block}{Remark}
Default value works well in most cases.
\end{block}

\end{frame}
%%%%%%%%%%%%%%%%%%%%%%%%%%%%%%%%%%%%%%%%%%%%%%%%%%%%%%%%%%%%%%%%%%%%%%%%%%%%%%%%%%%%%%%%%%%%%


%%%%%%%%%%%%%%%%%%%%%%%%%%%%%%%%%%%%%%%%%%%%%%%%%%%%%%%%%%%%%%%%%%%%%%%%%%%%%%%%%%%%%%%%%%%%%
\begin{frame}[allowframebreaks]{\texttt{L\_FINIT\_STP}} \label{L_FINIT_STP}
\vspace*{-12pt}
\begin{columns}
\column{0.4\linewidth}
\begin{block}{Type}
Double
\end{block}

\begin{block}{Default}
5e-3
\end{block}

\column{0.4\linewidth}
\begin{block}{Unit}
Bohr
\end{block}

\begin{block}{Example}
\texttt{L\_FINIT\_STP}: 0.01
\end{block}
\end{columns}

\begin{block}{Description}
Step length for line optimizer in LBFGS (\hyperlink{RELAX_METHOD}{\texttt{RELAX\_METHOD}}).
\end{block}

\begin{block}{Remark}
Default value works well in most cases.
\end{block}

\end{frame}
%%%%%%%%%%%%%%%%%%%%%%%%%%%%%%%%%%%%%%%%%%%%%%%%%%%%%%%%%%%%%%%%%%%%%%%%%%%%%%%%%%%%%%%%%%%%%


%%%%%%%%%%%%%%%%%%%%%%%%%%%%%%%%%%%%%%%%%%%%%%%%%%%%%%%%%%%%%%%%%%%%%%%%%%%%%%%%%%%%%%%%%%%%%
\begin{frame}[allowframebreaks]{\texttt{L\_MAXMOV}} \label{L_MAXMOV}
\vspace*{-12pt}
\begin{columns}
\column{0.4\linewidth}
\begin{block}{Type}
Double
\end{block}

\begin{block}{Default}
0.2
\end{block}

\column{0.4\linewidth}
\begin{block}{Unit}
Bohr
\end{block}

\begin{block}{Example}
\texttt{L\_MAXMOV}: 1.0
\end{block}
\end{columns}

\begin{block}{Description}
The maximum allowed step size in LBFGS (\hyperlink{RELAX_METHOD}{\texttt{RELAX\_METHOD}}).
\end{block}

\begin{block}{Remark}
Default value works well in most cases.
\end{block}

\end{frame}
%%%%%%%%%%%%%%%%%%%%%%%%%%%%%%%%%%%%%%%%%%%%%%%%%%%%%%%%%%%%%%%%%%%%%%%%%%%%%%%%%%%%%%%%%%%%%



%%%%%%%%%%%%%%%%%%%%%%%%%%%%%%%%%%%%%%%%%%%%%%%%%%%%%%%%%%%%%%%%%%%%%%%%%%%%%%%%%%%%%%%%%%%%%
\begin{frame}[allowframebreaks]{\texttt{L\_AUTOSCALE}} \label{L_AUTOSCALE}
\vspace*{-12pt}
\begin{columns}
\column{0.4\linewidth}
\begin{block}{Type}
Integer
\end{block}

\begin{block}{Default}
1
\end{block}

\column{0.4\linewidth}
\begin{block}{Unit}
No unit
\end{block}

\begin{block}{Example}
\texttt{L\_AUTOSCALE}: 0
\end{block}
\end{columns}

\begin{block}{Description}
Flag for automatically determining the inverse curvature that is used to determine the direction for next iteration in LBFGS (\hyperlink{RELAX_METHOD}{\texttt{RELAX\_METHOD}}).
\end{block}

\begin{block}{Remark}
Default works well in most cases.
\end{block}

\end{frame}
%%%%%%%%%%%%%%%%%%%%%%%%%%%%%%%%%%%%%%%%%%%%%%%%%%%%%%%%%%%%%%%%%%%%%%%%%%%%%%%%%%%%%%%%%%%%%

%%%%%%%%%%%%%%%%%%%%%%%%%%%%%%%%%%%%%%%%%%%%%%%%%%%%%%%%%%%%%%%%%%%%%%%%%%%%%%%%%%%%%%%%%%%%%
\begin{frame}[allowframebreaks]{\texttt{L\_LINEOPT}} \label{L_LINEOPT}
\vspace*{-12pt}
\begin{columns}
\column{0.4\linewidth}
\begin{block}{Type}
Integer
\end{block}

\begin{block}{Default}
1
\end{block}

\column{0.4\linewidth}
\begin{block}{Unit}
No unit
\end{block}

\begin{block}{Example}
\texttt{L\_LINEOPT}: 0
\end{block}
\end{columns}

\begin{block}{Description}
Flag for atomic force based line minimization in LBFGS (\hyperlink{RELAX_METHOD}{\texttt{RELAX\_METHOD}}). 
\end{block}

\begin{block}{Remark}
Required only if \hyperlink{L_AUTOSCALE}{\texttt{L\_AUTOSCALE}} is $0$.
\end{block}

\end{frame}
%%%%%%%%%%%%%%%%%%%%%%%%%%%%%%%%%%%%%%%%%%%%%%%%%%%%%%%%%%%%%%%%%%%%%%%%%%%%%%%%%%%%%%%%%%%%%


%%%%%%%%%%%%%%%%%%%%%%%%%%%%%%%%%%%%%%%%%%%%%%%%%%%%%%%%%%%%%%%%%%%%%%%%%%%%%%%%%%%%%%%%%%%%%
\begin{frame}[allowframebreaks]{\texttt{L\_ICURV}} \label{L_ICURV}
\vspace*{-12pt}
\begin{columns}
\column{0.4\linewidth}
\begin{block}{Type}
Double
\end{block}

\begin{block}{Default}
1.0
\end{block}

\column{0.4\linewidth}
\begin{block}{Unit}
No unit
\end{block}

\begin{block}{Example}
\texttt{L\_ICURV}: 0.1
\end{block}
\end{columns}

\begin{block}{Description}
Initial inverse curvature, used to construct the inverse Hessian matrix in LBFGS (\hyperlink{RELAX_METHOD}{\texttt{RELAX\_METHOD}}). 
\end{block}

\begin{block}{Remark}
Needed only if  \hyperlink{L_AUTOSCALE}{\texttt{L\_AUTOSCALE}} is $0$. Default value works well in most cases.
\end{block}

\end{frame}
%%%%%%%%%%%%%%%%%%%%%%%%%%%%%%%%%%%%%%%%%%%%%%%%%%%%%%%%%%%%%%%%%%%%%%%%%%%%%%%%%%%%%%%%%%%%%

%%%%%%%%%%%%%%%%%%%%%%%%%%%%%%%%%%%%%%%%%%%%%%%%%%%%%%%%%%%%%%%%%%%%%%%%%%%%%%%%%%%%%%%%%%%%%
\begin{frame}[allowframebreaks]{\texttt{FIRE\_DT}} \label{FIRE_DT}
\vspace*{-12pt}
\begin{columns}
\column{0.4\linewidth}
\begin{block}{Type}
Double
\end{block}

\begin{block}{Default}
1
\end{block}

\column{0.4\linewidth}
\begin{block}{Unit}
Femto second
\end{block}

\begin{block}{Example}
\texttt{FIRE\_DT}: 0.1
\end{block}
\end{columns}

\begin{block}{Description}
Time step used in FIRE (\hyperlink{RELAX_METHOD}{\texttt{RELAX\_METHOD}}).
\end{block}

\begin{block}{Remark}
Default value works well in most cases.
\end{block}

\end{frame}
%%%%%%%%%%%%%%%%%%%%%%%%%%%%%%%%%%%%%%%%%%%%%%%%%%%%%%%%%%%%%%%%%%%%%%%%%%%%%%%%%%%%%%%%%%%%%


%%%%%%%%%%%%%%%%%%%%%%%%%%%%%%%%%%%%%%%%%%%%%%%%%%%%%%%%%%%%%%%%%%%%%%%%%%%%%%%%%%%%%%%%%%%%%
\begin{frame}[allowframebreaks]{\texttt{FIRE\_MASS}} \label{FIRE_MASS}
\vspace*{-12pt}
\begin{columns}
\column{0.4\linewidth}
\begin{block}{Type}
Double
\end{block}

\begin{block}{Default}
1.0
\end{block}

\column{0.4\linewidth}
\begin{block}{Unit}
Atomic mass unit
\end{block}

\begin{block}{Example}
\texttt{FIRE\_MASS}: 2.5
\end{block}
\end{columns}

\begin{block}{Description}
Pseudomass used in FIRE (\hyperlink{RELAX_METHOD}{\texttt{RELAX\_METHOD}}).
\end{block}

\begin{block}{Remark}
Default value works well in most cases.
\end{block}

\end{frame}
%%%%%%%%%%%%%%%%%%%%%%%%%%%%%%%%%%%%%%%%%%%%%%%%%%%%%%%%%%%%%%%%%%%%%%%%%%%%%%%%%%%%%%%%%%%%%



%%%%%%%%%%%%%%%%%%%%%%%%%%%%%%%%%%%%%%%%%%%%%%%%%%%%%%%%%%%%%%%%%%%%%%%%%%%%%%%%%%%%%%%%%%%%%
\begin{frame}[allowframebreaks]{\texttt{FIRE\_MAXMOV}} \label{FIRE_MAXMOV}
\vspace*{-12pt}
\begin{columns}
\column{0.4\linewidth}
\begin{block}{Type}
Double
\end{block}

\begin{block}{Default}
0.2
\end{block}

\column{0.4\linewidth}
\begin{block}{Unit}
Bohr
\end{block}

\begin{block}{Example}
\texttt{FIRE\_MAXMOV}: 1.0
\end{block}
\end{columns}

\begin{block}{Description}
Maximum movement for any atom in FIRE (\hyperlink{RELAX_METHOD}{\texttt{RELAX\_METHOD}}).
\end{block}

\begin{block}{Remark}
Default value works well in most cases.
\end{block}

\end{frame}
%%%%%%%%%%%%%%%%%%%%%%%%%%%%%%%%%%%%%%%%%%%%%%%%%%%%%%%%%%%%%%%%%%%%%%%%%%%%%%%%%%%%%%%%%%%%%


%%%%%%%%%%%%%%%%%%%%%%%%%%%%%%%%%%%%%%%%%%%%%%%%%%%%%%%%%%%%%%%%%%%%%%%%%%%%%%%%%%%%%%%%%%%%%
\begin{frame}[allowframebreaks,c]{} \label{Print options}

\begin{center}
\Huge \textbf{Print options}
\end{center}

\end{frame}
%%%%%%%%%%%%%%%%%%%%%%%%%%%%%%%%%%%%%%%%%%%%%%%%%%%%%%%%%%%%%%%%%%%%%%%%%%%%%%%%%%%%%%%%%%%%%




%%%%%%%%%%%%%%%%%%%%%%%%%%%%%%%%%%%%%%%%%%%%%%%%%%%%%%%%%%%%%%%%%%%%%%%%%%%%%%%%%%%%%%%%%%%%%
\begin{frame}[allowframebreaks]{\texttt{PRINT\_ATOMS}} \label{PRINT_ATOMS}
\vspace*{-12pt}
\begin{columns}
\column{0.4\linewidth}
\begin{block}{Type}
0 or 1
\end{block}

\begin{block}{Default}
0
\end{block}

\column{0.4\linewidth}
\begin{block}{Unit}
No unit
\end{block}

\begin{block}{Example}
\texttt{PRINT\_ATOMS}: 1
\end{block}
\end{columns}

\begin{block}{Description}
Flag for writing the atomic positions. For ground-state calculations, atom positions are printed to a `.static' output file. For structural relaxation calculations, atom positions are printed to a `.geopt' file. For MD calculations, atom positions are printed to a `.aimd' file.
\end{block}


\end{frame}
%%%%%%%%%%%%%%%%%%%%%%%%%%%%%%%%%%%%%%%%%%%%%%%%%%%%%%%%%%%%%%%%%%%%%%%%%%%%%%%%%%%%%%%%%%%%%




%%%%%%%%%%%%%%%%%%%%%%%%%%%%%%%%%%%%%%%%%%%%%%%%%%%%%%%%%%%%%%%%%%%%%%%%%%%%%%%%%%%%%%%%%%%%%
\begin{frame}[allowframebreaks]{\texttt{PRINT\_FORCES}} \label{PRINT_FORCES}
\vspace*{-12pt}
\begin{columns}
\column{0.4\linewidth}
\begin{block}{Type}
0 or 1
\end{block}

\begin{block}{Default}
0
\end{block}

\column{0.4\linewidth}
\begin{block}{Unit}
No unit
\end{block}

\begin{block}{Example}
\texttt{PRINT\_FORCES}: 1
\end{block}
\end{columns}

\begin{block}{Description}
Flag for writing the atomic forces. For ground-state calculations, forces are printed to a `.static' output file. For structural relaxation calculations, forces are printed to a `.geopt' file. For MD calculations, forces are printed to a `.aimd' file.
\end{block}

\end{frame}
%%%%%%%%%%%%%%%%%%%%%%%%%%%%%%%%%%%%%%%%%%%%%%%%%%%%%%%%%%%%%%%%%%%%%%%%%%%%%%%%%%%%%%%%%%%%%




%%%%%%%%%%%%%%%%%%%%%%%%%%%%%%%%%%%%%%%%%%%%%%%%%%%%%%%%%%%%%%%%%%%%%%%%%%%%%%%%%%%%%%%%%%%%%
\begin{frame}[allowframebreaks]{\texttt{PRINT\_MDOUT}} \label{PRINT_MDOUT}
\vspace*{-12pt}
\begin{columns}
\column{0.4\linewidth}
\begin{block}{Type}
0 or 1
\end{block}

\begin{block}{Default}
1
\end{block}

\column{0.4\linewidth}
\begin{block}{Unit}
No unit
\end{block}

\begin{block}{Example}
\texttt{PRINT\_MDOUT}: 0
\end{block}
\end{columns}

\begin{block}{Description}
Flag for printing the the MD output into the .aimd file. 
\end{block}

%\begin{block}{Remark}
%\end{block}

\end{frame}
%%%%%%%%%%%%%%%%%%%%%%%%%%%%%%%%%%%%%%%%%%%%%%%%%%%%%%%%%%%%%%%%%%%%%%%%%%%%%%%%%%%%%%%%%%%%%





%%%%%%%%%%%%%%%%%%%%%%%%%%%%%%%%%%%%%%%%%%%%%%%%%%%%%%%%%%%%%%%%%%%%%%%%%%%%%%%%%%%%%%%%%%%%%
\begin{frame}[allowframebreaks]{\texttt{PRINT\_RELAXOUT}} \label{PRINT_RELAXOUT}
\vspace*{-12pt}
\begin{columns}
\column{0.4\linewidth}
\begin{block}{Type}
0 or 1
\end{block}

\begin{block}{Default}
1
\end{block}

\column{0.4\linewidth}
\begin{block}{Unit}
No unit
\end{block}

\begin{block}{Example}
\texttt{PRINT\_RELAXOUT}: 0
\end{block}
\end{columns}

\begin{block}{Description}
Flag for printing the structural relaxation data in a .geopt file.
\end{block}

\begin{block}{Remark}
Required only if \hyperlink{RELAX_FLAG}{\texttt{RELAX\_FLAG}} is set to $1$.
\end{block}

\end{frame}
%%%%%%%%%%%%%%%%%%%%%%%%%%%%%%%%%%%%%%%%%%%%%%%%%%%%%%%%%%%%%%%%%%%%%%%%%%%%%%%%%%%%%%%%%%%%%



%%%%%%%%%%%%%%%%%%%%%%%%%%%%%%%%%%%%%%%%%%%%%%%%%%%%%%%%%%%%%%%%%%%%%%%%%%%%%%%%%%%%%%%%%%%%%
\begin{frame}[allowframebreaks]{\texttt{PRINT\_RESTART}} \label{PRINT_RESTART}
\vspace*{-12pt}
\begin{columns}
\column{0.4\linewidth}
\begin{block}{Type}
0 or 1
\end{block}

\begin{block}{Default}
1
\end{block}

\column{0.4\linewidth}
\begin{block}{Unit}
No unit
\end{block}

\begin{block}{Example}
\texttt{PRINT\_RESTART}: 0
\end{block}
\end{columns}

\begin{block}{Description}
Flag for writing the .restart file, used to restart MD and structural relaxation simulations. 
\end{block}

\begin{block}{Remark}
Relevant only if either \hyperlink{MD_FLAG}{\texttt{MD\_FLAG}} is $1$ or \hyperlink{RELAX_FLAG}{\texttt{RELAX\_FLAG}} is $1$.
\end{block}

\end{frame}
%%%%%%%%%%%%%%%%%%%%%%%%%%%%%%%%%%%%%%%%%%%%%%%%%%%%%%%%%%%%%%%%%%%%%%%%%%%%%%%%%%%%%%%%%%%%%




%%%%%%%%%%%%%%%%%%%%%%%%%%%%%%%%%%%%%%%%%%%%%%%%%%%%%%%%%%%%%%%%%%%%%%%%%%%%%%%%%%%%%%%%%%%%%
\begin{frame}[allowframebreaks]{\texttt{PRINT\_RESTART\_FQ}} \label{PRINT_RESTART_FQ}
\vspace*{-12pt}
\begin{columns}
\column{0.4\linewidth}
\begin{block}{Type}
Integer
\end{block}

\begin{block}{Default}
1
\end{block}

\column{0.4\linewidth}
\begin{block}{Unit}
No unit
\end{block}

\begin{block}{Example}
\texttt{PRINT\_RESTART\_FQ}: 10
\end{block}
\end{columns}

\begin{block}{Description}
Frequency at which .restart file is written in MD and structural optimization simulations.
\end{block}

\begin{block}{Remark}
Relevant only if either \hyperlink{MD_FLAG}{\texttt{MD\_FLAG}} is $1$ or \hyperlink{RELAX_FLAG}{\texttt{RELAX\_FLAG}} is $1$.
\end{block}

\end{frame}
%%%%%%%%%%%%%%%%%%%%%%%%%%%%%%%%%%%%%%%%%%%%%%%%%%%%%%%%%%%%%%%%%%%%%%%%%%%%%%%%%%%%%%%%%%%%%




%%%%%%%%%%%%%%%%%%%%%%%%%%%%%%%%%%%%%%%%%%%%%%%%%%%%%%%%%%%%%%%%%%%%%%%%%%%%%%%%%%%%%%%%%%%%%
\begin{frame}[allowframebreaks]{\texttt{PRINT\_VELS}} \label{PRINT_VELS}
\vspace*{-12pt}
\begin{columns}
\column{0.4\linewidth}
\begin{block}{Type}
0 or 1
\end{block}

\begin{block}{Default}
1
\end{block}

\column{0.4\linewidth}
\begin{block}{Unit}
No unit
\end{block}

\begin{block}{Example}
\texttt{PRINT\_VELS}: 0
\end{block}
\end{columns}

\begin{block}{Description}
Flag for printing the ion velocities in an MD simulation into the .aimd file. 
\end{block}

\begin{block}{Remark}
Relevant only if \hyperlink{MD_FLAG}{\texttt{MD\_FLAG}} is set to $1$.
\end{block}

\end{frame}
%%%%%%%%%%%%%%%%%%%%%%%%%%%%%%%%%%%%%%%%%%%%%%%%%%%%%%%%%%%%%%%%%%%%%%%%%%%%%%%%%%%%%%%%%%%%%




%%%%%%%%%%%%%%%%%%%%%%%%%%%%%%%%%%%%%%%%%%%%%%%%%%%%%%%%%%%%%%%%%%%%%%%%%%%%%%%%%%%%%%%%%%%%%
\begin{frame}[allowframebreaks]{\texttt{OUTPUT\_FILE}} \label{OUTPUT_FILE}
\vspace*{-12pt}
\begin{columns}
\column{0.4\linewidth}
\begin{block}{Type}
String
\end{block}

\begin{block}{Default}
Same as the input file name
\end{block}

\column{0.4\linewidth}
\begin{block}{Unit}
No unit
\end{block}

\begin{block}{Example}
\texttt{OUTPUT\_FILE}: myfname
\end{block}
\end{columns}

\begin{block}{Description}
The name of the output files. The output files are attached with a suffix (`.out',`.static',`.geopt' and `.aimd'). 
\end{block}

\begin{block}{Remark}
If an output file with the same name already exist, the results will be written to a file with a number attached, e.g., `myfname.out\_1'. The maximum number of output files with the same name allowed is 100. After that the output files will be overwritten in succession.
\end{block}

\end{frame}
%%%%%%%%%%%%%%%%%%%%%%%%%%%%%%%%%%%%%%%%%%%%%%%%%%%%%%%%%%%%%%%%%%%%%%%%%%%%%%%%%%%%%%%%%%%%%



\end{document}

